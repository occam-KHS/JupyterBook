%% Generated by Sphinx.
\def\sphinxdocclass{jupyterBook}
\documentclass[letterpaper,10pt,english]{jupyterBook}
\ifdefined\pdfpxdimen
   \let\sphinxpxdimen\pdfpxdimen\else\newdimen\sphinxpxdimen
\fi \sphinxpxdimen=.75bp\relax
\ifdefined\pdfimageresolution
    \pdfimageresolution= \numexpr \dimexpr1in\relax/\sphinxpxdimen\relax
\fi
%% let collapsible pdf bookmarks panel have high depth per default
\PassOptionsToPackage{bookmarksdepth=5}{hyperref}
%% turn off hyperref patch of \index as sphinx.xdy xindy module takes care of
%% suitable \hyperpage mark-up, working around hyperref-xindy incompatibility
\PassOptionsToPackage{hyperindex=false}{hyperref}
%% memoir class requires extra handling
\makeatletter\@ifclassloaded{memoir}
{\ifdefined\memhyperindexfalse\memhyperindexfalse\fi}{}\makeatother

\PassOptionsToPackage{warn}{textcomp}

\catcode`^^^^00a0\active\protected\def^^^^00a0{\leavevmode\nobreak\ }
\usepackage{cmap}
\usepackage{fontspec}
\defaultfontfeatures[\rmfamily,\sffamily,\ttfamily]{}
\usepackage{amsmath,amssymb,amstext}
\usepackage{polyglossia}
\setmainlanguage{english}



\setmainfont{FreeSerif}[
  Extension      = .otf,
  UprightFont    = *,
  ItalicFont     = *Italic,
  BoldFont       = *Bold,
  BoldItalicFont = *BoldItalic
]
\setsansfont{FreeSans}[
  Extension      = .otf,
  UprightFont    = *,
  ItalicFont     = *Oblique,
  BoldFont       = *Bold,
  BoldItalicFont = *BoldOblique,
]
\setmonofont{FreeMono}[
  Extension      = .otf,
  UprightFont    = *,
  ItalicFont     = *Oblique,
  BoldFont       = *Bold,
  BoldItalicFont = *BoldOblique,
]



\usepackage[Bjarne]{fncychap}
\usepackage[,numfigreset=1,mathnumfig]{sphinx}

\fvset{fontsize=\small}
\usepackage{geometry}


% Include hyperref last.
\usepackage{hyperref}
% Fix anchor placement for figures with captions.
\usepackage{hypcap}% it must be loaded after hyperref.
% Set up styles of URL: it should be placed after hyperref.
\urlstyle{same}


\usepackage{sphinxmessages}



        % Start of preamble defined in sphinx-jupyterbook-latex %
         \usepackage[Latin,Greek]{ucharclasses}
        \usepackage{unicode-math}
        % fixing title of the toc
        \addto\captionsenglish{\renewcommand{\contentsname}{Contents}}
        \hypersetup{
            pdfencoding=auto,
            psdextra
        }
        % End of preamble defined in sphinx-jupyterbook-latex %
        

\title{Resample}
\date{Jul 02, 2022}
\release{}
\author{KHS}
\newcommand{\sphinxlogo}{\vbox{}}
\renewcommand{\releasename}{}
\makeindex
\begin{document}

\pagestyle{empty}
\sphinxmaketitle
\pagestyle{plain}
\sphinxtableofcontents
\pagestyle{normal}
\phantomsection\label{\detokenize{chapter2/2.2.4_Useful_Techniques::doc}}


\sphinxAtStartPar
Resample 은 시간데이터를 다른 시간 단위로 변경하고 싶을 때 활용합니다. 예를 들면, 초 단위 데이터를 일단위 혹은 월단위 데이터로 변경 할 수 있습니다. 연습을 위하여 시간 레벨의 데이터가 필요합니다. 시간레벨 데이터는 FinanceDataReader 패키지에서 제공하는 일봉 데이터를 활용하겠습니다. FinanceDataReader 는 이승준님이 금융자료 분석을 하시는 분들을 위하여 만들어 주신 정말 유용한 패키지입니다. 자세한 내용은 아래 링크에 설명이 되어 있습니다.
\sphinxurl{https://financedata.github.io/posts/finance-data-reader-users-guide.html} 또한, 이승준님이 Pycon 에서 엑셀에 비하여 파이썬의 장점에 대하여 강연하시는 내용이 유투브에 있습니다. \sphinxurl{https://www.youtube.com/watch?v=w7Q\_eKN5r-I}


\part{FinanceDataReader}
\label{\detokenize{chapter2/2.2.4_Useful_Techniques:financedatareader}}
\sphinxAtStartPar
FinanceDataReader 를 import 합니다. DataReader 함수에 종목코드, 시작일, 종료일을 인수로 넣어주면 아래와 같이 일봉데이터를 리턴합니다. 출력해보면 Date 가 index 로 되어 있음을 알 수 있습니다.

\begin{sphinxuseclass}{cell}\begin{sphinxVerbatimInput}

\begin{sphinxuseclass}{cell_input}
\begin{sphinxVerbatim}[commandchars=\\\{\}]
\PYG{k+kn}{import} \PYG{n+nn}{FinanceDataReader} \PYG{k}{as} \PYG{n+nn}{fdr}

\PYG{n}{code} \PYG{o}{=} \PYG{l+s+s1}{\PYGZsq{}}\PYG{l+s+s1}{005930}\PYG{l+s+s1}{\PYGZsq{}} \PYG{c+c1}{\PYGZsh{} 삼성전자}
\PYG{n}{stock\PYGZus{}data} \PYG{o}{=} \PYG{n}{fdr}\PYG{o}{.}\PYG{n}{DataReader}\PYG{p}{(}\PYG{n}{code}\PYG{p}{,} \PYG{n}{start}\PYG{o}{=}\PYG{l+s+s1}{\PYGZsq{}}\PYG{l+s+s1}{2021\PYGZhy{}01\PYGZhy{}03}\PYG{l+s+s1}{\PYGZsq{}}\PYG{p}{,} \PYG{n}{end}\PYG{o}{=}\PYG{l+s+s1}{\PYGZsq{}}\PYG{l+s+s1}{2021\PYGZhy{}12\PYGZhy{}31}\PYG{l+s+s1}{\PYGZsq{}}\PYG{p}{)} 

\PYG{n}{stock\PYGZus{}data}\PYG{o}{.}\PYG{n}{head}\PYG{p}{(}\PYG{p}{)}\PYG{o}{.}\PYG{n}{style}\PYG{o}{.}\PYG{n}{set\PYGZus{}table\PYGZus{}attributes}\PYG{p}{(}\PYG{l+s+s1}{\PYGZsq{}}\PYG{l+s+s1}{style=}\PYG{l+s+s1}{\PYGZdq{}}\PYG{l+s+s1}{font\PYGZhy{}size: 12px}\PYG{l+s+s1}{\PYGZdq{}}\PYG{l+s+s1}{\PYGZsq{}}\PYG{p}{)} \PYG{c+c1}{\PYGZsh{} head 메소드는 처음 5 row 만 출력합니다.}
\end{sphinxVerbatim}

\end{sphinxuseclass}\end{sphinxVerbatimInput}
\begin{sphinxVerbatimOutput}

\begin{sphinxuseclass}{cell_output}
\begin{sphinxVerbatim}[commandchars=\\\{\}]
\PYGZlt{}pandas.io.formats.style.Styler at 0x1eab9282310\PYGZgt{}
\end{sphinxVerbatim}

\end{sphinxuseclass}\end{sphinxVerbatimOutput}

\end{sphinxuseclass}


\begin{sphinxuseclass}{cell}\begin{sphinxVerbatimInput}

\begin{sphinxuseclass}{cell_input}
\begin{sphinxVerbatim}[commandchars=\\\{\}]
\PYG{k+kn}{import} \PYG{n+nn}{pandas} \PYG{k}{as} \PYG{n+nn}{pd}
\PYG{n}{pd}\PYG{o}{.}\PYG{n}{options}\PYG{o}{.}\PYG{n}{display}\PYG{o}{.}\PYG{n}{float\PYGZus{}format} \PYG{o}{=} \PYG{l+s+s1}{\PYGZsq{}}\PYG{l+s+si}{\PYGZob{}:,.0f\PYGZcb{}}\PYG{l+s+s1}{\PYGZsq{}}\PYG{o}{.}\PYG{n}{format}
\PYG{n}{stock\PYGZus{}data}\PYG{o}{.}\PYG{n}{resample}\PYG{p}{(}\PYG{l+s+s1}{\PYGZsq{}}\PYG{l+s+s1}{M}\PYG{l+s+s1}{\PYGZsq{}}\PYG{p}{)}\PYG{p}{[}\PYG{l+s+s1}{\PYGZsq{}}\PYG{l+s+s1}{Close}\PYG{l+s+s1}{\PYGZsq{}}\PYG{p}{]}\PYG{o}{.}\PYG{n}{agg}\PYG{p}{(}\PYG{p}{[}\PYG{l+s+s1}{\PYGZsq{}}\PYG{l+s+s1}{mean}\PYG{l+s+s1}{\PYGZsq{}}\PYG{p}{,}\PYG{l+s+s1}{\PYGZsq{}}\PYG{l+s+s1}{max}\PYG{l+s+s1}{\PYGZsq{}}\PYG{p}{,}\PYG{l+s+s1}{\PYGZsq{}}\PYG{l+s+s1}{min}\PYG{l+s+s1}{\PYGZsq{}}\PYG{p}{]}\PYG{p}{)}\PYG{o}{.}\PYG{n}{head}\PYG{p}{(}\PYG{p}{)}\PYG{o}{.}\PYG{n}{style}\PYG{o}{.}\PYG{n}{set\PYGZus{}table\PYGZus{}attributes}\PYG{p}{(}\PYG{l+s+s1}{\PYGZsq{}}\PYG{l+s+s1}{style=}\PYG{l+s+s1}{\PYGZdq{}}\PYG{l+s+s1}{font\PYGZhy{}size: 12px}\PYG{l+s+s1}{\PYGZdq{}}\PYG{l+s+s1}{\PYGZsq{}}\PYG{p}{)} \PYG{c+c1}{\PYGZsh{} 처음 5개만 출력}
\end{sphinxVerbatim}

\end{sphinxuseclass}\end{sphinxVerbatimInput}
\begin{sphinxVerbatimOutput}

\begin{sphinxuseclass}{cell_output}
\begin{sphinxVerbatim}[commandchars=\\\{\}]
\PYGZlt{}pandas.io.formats.style.Styler at 0x1eab69f9af0\PYGZgt{}
\end{sphinxVerbatim}

\end{sphinxuseclass}\end{sphinxVerbatimOutput}

\end{sphinxuseclass}
\sphinxAtStartPar
주별로 요약할 수 도 있습니다. 이번에는 resample(‘W’) 라고 해 줍니다. Resample 이 정말 유용한 기능이라는 것을 직감하셨을 것으로 생각합니다. 역시 한 주(월요일 \textasciitilde{} 일요일)의 마지막날이 Index 로 들어가 있습니다. 디폴트는 일요일입니다.

\begin{sphinxuseclass}{cell}\begin{sphinxVerbatimInput}

\begin{sphinxuseclass}{cell_input}
\begin{sphinxVerbatim}[commandchars=\\\{\}]
\PYG{n}{pd}\PYG{o}{.}\PYG{n}{options}\PYG{o}{.}\PYG{n}{display}\PYG{o}{.}\PYG{n}{float\PYGZus{}format} \PYG{o}{=} \PYG{l+s+s1}{\PYGZsq{}}\PYG{l+s+si}{\PYGZob{}:,.0f\PYGZcb{}}\PYG{l+s+s1}{\PYGZsq{}}\PYG{o}{.}\PYG{n}{format}
\PYG{n}{stock\PYGZus{}data}\PYG{o}{.}\PYG{n}{resample}\PYG{p}{(}\PYG{l+s+s1}{\PYGZsq{}}\PYG{l+s+s1}{W}\PYG{l+s+s1}{\PYGZsq{}}\PYG{p}{)}\PYG{p}{[}\PYG{l+s+s1}{\PYGZsq{}}\PYG{l+s+s1}{Close}\PYG{l+s+s1}{\PYGZsq{}}\PYG{p}{]}\PYG{o}{.}\PYG{n}{agg}\PYG{p}{(}\PYG{p}{[}\PYG{l+s+s1}{\PYGZsq{}}\PYG{l+s+s1}{mean}\PYG{l+s+s1}{\PYGZsq{}}\PYG{p}{,}\PYG{l+s+s1}{\PYGZsq{}}\PYG{l+s+s1}{max}\PYG{l+s+s1}{\PYGZsq{}}\PYG{p}{,}\PYG{l+s+s1}{\PYGZsq{}}\PYG{l+s+s1}{min}\PYG{l+s+s1}{\PYGZsq{}}\PYG{p}{]}\PYG{p}{)}\PYG{o}{.}\PYG{n}{head}\PYG{p}{(}\PYG{p}{)}\PYG{o}{.}\PYG{n}{style}\PYG{o}{.}\PYG{n}{set\PYGZus{}table\PYGZus{}attributes}\PYG{p}{(}\PYG{l+s+s1}{\PYGZsq{}}\PYG{l+s+s1}{style=}\PYG{l+s+s1}{\PYGZdq{}}\PYG{l+s+s1}{font\PYGZhy{}size: 12px}\PYG{l+s+s1}{\PYGZdq{}}\PYG{l+s+s1}{\PYGZsq{}}\PYG{p}{)}
\end{sphinxVerbatim}

\end{sphinxuseclass}\end{sphinxVerbatimInput}
\begin{sphinxVerbatimOutput}

\begin{sphinxuseclass}{cell_output}
\begin{sphinxVerbatim}[commandchars=\\\{\}]
\PYGZlt{}pandas.io.formats.style.Styler at 0x1eab947fdc0\PYGZgt{}
\end{sphinxVerbatim}

\end{sphinxuseclass}\end{sphinxVerbatimOutput}

\end{sphinxuseclass}






\renewcommand{\indexname}{Index}
\printindex
\end{document}