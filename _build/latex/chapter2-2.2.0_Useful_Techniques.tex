%% Generated by Sphinx.
\def\sphinxdocclass{jupyterBook}
\documentclass[letterpaper,10pt,english]{jupyterBook}
\ifdefined\pdfpxdimen
   \let\sphinxpxdimen\pdfpxdimen\else\newdimen\sphinxpxdimen
\fi \sphinxpxdimen=.75bp\relax
\ifdefined\pdfimageresolution
    \pdfimageresolution= \numexpr \dimexpr1in\relax/\sphinxpxdimen\relax
\fi
%% let collapsible pdf bookmarks panel have high depth per default
\PassOptionsToPackage{bookmarksdepth=5}{hyperref}
%% turn off hyperref patch of \index as sphinx.xdy xindy module takes care of
%% suitable \hyperpage mark-up, working around hyperref-xindy incompatibility
\PassOptionsToPackage{hyperindex=false}{hyperref}
%% memoir class requires extra handling
\makeatletter\@ifclassloaded{memoir}
{\ifdefined\memhyperindexfalse\memhyperindexfalse\fi}{}\makeatother

\PassOptionsToPackage{warn}{textcomp}

\catcode`^^^^00a0\active\protected\def^^^^00a0{\leavevmode\nobreak\ }
\usepackage{cmap}
\usepackage{fontspec}
\defaultfontfeatures[\rmfamily,\sffamily,\ttfamily]{}
\usepackage{amsmath,amssymb,amstext}
\usepackage{polyglossia}
\setmainlanguage{english}



\setmainfont{FreeSerif}[
  Extension      = .otf,
  UprightFont    = *,
  ItalicFont     = *Italic,
  BoldFont       = *Bold,
  BoldItalicFont = *BoldItalic
]
\setsansfont{FreeSans}[
  Extension      = .otf,
  UprightFont    = *,
  ItalicFont     = *Oblique,
  BoldFont       = *Bold,
  BoldItalicFont = *BoldOblique,
]
\setmonofont{FreeMono}[
  Extension      = .otf,
  UprightFont    = *,
  ItalicFont     = *Oblique,
  BoldFont       = *Bold,
  BoldItalicFont = *BoldOblique,
]



\usepackage[Bjarne]{fncychap}
\usepackage[,numfigreset=1,mathnumfig]{sphinx}

\fvset{fontsize=\small}
\usepackage{geometry}


% Include hyperref last.
\usepackage{hyperref}
% Fix anchor placement for figures with captions.
\usepackage{hypcap}% it must be loaded after hyperref.
% Set up styles of URL: it should be placed after hyperref.
\urlstyle{same}


\usepackage{sphinxmessages}



        % Start of preamble defined in sphinx-jupyterbook-latex %
         \usepackage[Latin,Greek]{ucharclasses}
        \usepackage{unicode-math}
        % fixing title of the toc
        \addto\captionsenglish{\renewcommand{\contentsname}{Contents}}
        \hypersetup{
            pdfencoding=auto,
            psdextra
        }
        % End of preamble defined in sphinx-jupyterbook-latex %
        

\title{유용한 기능들}
\date{Jul 02, 2022}
\release{}
\author{KHS}
\newcommand{\sphinxlogo}{\vbox{}}
\renewcommand{\releasename}{}
\makeindex
\begin{document}

\pagestyle{empty}
\sphinxmaketitle
\pagestyle{plain}
\sphinxtableofcontents
\pagestyle{normal}
\phantomsection\label{\detokenize{chapter2/2.2.0_Useful_Techniques::doc}}


\sphinxAtStartPar
이번 장에서는 유용한 데이터 핸들링 방법들을 배우겠습니다. 데이터 분석을 통해서 원하는 결과를 얻기 위해서는, 우선 분석이 가능한 형태의 데이터로 가공을 해야합니다.


\part{Append}
\label{\detokenize{chapter2/2.2.1_Useful_Techniques:append}}\label{\detokenize{chapter2/2.2.1_Useful_Techniques::doc}}
\sphinxAtStartPar
append 는 반복문에서 발생하는 값을 순차적으로 모으는데 유용합니다. 아래 예제는 반복문에서 추출된 원소를 제곱한 값을 계속 v\_list 리스트에 추가하는 코드입니다.

\begin{sphinxuseclass}{cell}\begin{sphinxVerbatimInput}

\begin{sphinxuseclass}{cell_input}
\begin{sphinxVerbatim}[commandchars=\\\{\}]
\PYG{n}{v\PYGZus{}list} \PYG{o}{=} \PYG{p}{[}\PYG{p}{]}

\PYG{n}{aa} \PYG{o}{=} \PYG{p}{[}\PYG{l+m+mi}{1}\PYG{p}{,} \PYG{l+m+mi}{2}\PYG{p}{,} \PYG{l+m+mi}{3}\PYG{p}{,} \PYG{l+m+mi}{4}\PYG{p}{,} \PYG{l+m+mi}{5}\PYG{p}{]}

\PYG{k}{for} \PYG{n}{a} \PYG{o+ow}{in} \PYG{n}{aa}\PYG{p}{:}
    \PYG{n}{v\PYGZus{}list}\PYG{o}{.}\PYG{n}{append}\PYG{p}{(}\PYG{n}{a}\PYG{o}{*}\PYG{o}{*}\PYG{l+m+mi}{2}\PYG{p}{)}
    
\PYG{n+nb}{print}\PYG{p}{(}\PYG{n}{v\PYGZus{}list}\PYG{p}{)}    
\end{sphinxVerbatim}

\end{sphinxuseclass}\end{sphinxVerbatimInput}
\begin{sphinxVerbatimOutput}

\begin{sphinxuseclass}{cell_output}
\begin{sphinxVerbatim}[commandchars=\\\{\}]
[1, 4, 9, 16, 25]
\end{sphinxVerbatim}

\end{sphinxuseclass}\end{sphinxVerbatimOutput}

\end{sphinxuseclass}


\begin{sphinxuseclass}{cell}\begin{sphinxVerbatimInput}

\begin{sphinxuseclass}{cell_input}
\begin{sphinxVerbatim}[commandchars=\\\{\}]
\PYG{k+kn}{import} \PYG{n+nn}{pandas} \PYG{k}{as} \PYG{n+nn}{pd}

\PYG{n}{r\PYGZus{}list} \PYG{o}{=} \PYG{p}{[}\PYG{p}{]}

\PYG{n}{c1\PYGZus{}list} \PYG{o}{=} \PYG{p}{[}\PYG{l+m+mi}{11}\PYG{p}{,}\PYG{l+m+mi}{12}\PYG{p}{,}\PYG{l+m+mi}{13}\PYG{p}{,}\PYG{l+m+mi}{14}\PYG{p}{,}\PYG{l+m+mi}{15}\PYG{p}{]}
\PYG{n}{c2\PYGZus{}list} \PYG{o}{=} \PYG{p}{[}\PYG{l+s+s1}{\PYGZsq{}}\PYG{l+s+s1}{a}\PYG{l+s+s1}{\PYGZsq{}}\PYG{p}{,}\PYG{l+s+s1}{\PYGZsq{}}\PYG{l+s+s1}{b}\PYG{l+s+s1}{\PYGZsq{}}\PYG{p}{,}\PYG{l+s+s1}{\PYGZsq{}}\PYG{l+s+s1}{c}\PYG{l+s+s1}{\PYGZsq{}}\PYG{p}{,}\PYG{l+s+s1}{\PYGZsq{}}\PYG{l+s+s1}{d}\PYG{l+s+s1}{\PYGZsq{}}\PYG{p}{,}\PYG{l+s+s1}{\PYGZsq{}}\PYG{l+s+s1}{e}\PYG{l+s+s1}{\PYGZsq{}}\PYG{p}{]}

\PYG{n}{df1} \PYG{o}{=} \PYG{n}{pd}\PYG{o}{.}\PYG{n}{DataFrame}\PYG{p}{(}\PYG{p}{\PYGZob{}}\PYG{l+s+s1}{\PYGZsq{}}\PYG{l+s+s1}{c1}\PYG{l+s+s1}{\PYGZsq{}}\PYG{p}{:} \PYG{n}{c1\PYGZus{}list}\PYG{p}{,} \PYG{l+s+s1}{\PYGZsq{}}\PYG{l+s+s1}{c2}\PYG{l+s+s1}{\PYGZsq{}}\PYG{p}{:} \PYG{n}{c2\PYGZus{}list}\PYG{p}{\PYGZcb{}}\PYG{p}{)}

\PYG{k}{for} \PYG{n}{i} \PYG{o+ow}{in} \PYG{n+nb}{list}\PYG{p}{(}\PYG{n}{df1}\PYG{p}{[}\PYG{l+s+s1}{\PYGZsq{}}\PYG{l+s+s1}{c1}\PYG{l+s+s1}{\PYGZsq{}}\PYG{p}{]}\PYG{p}{)}\PYG{p}{:} \PYG{c+c1}{\PYGZsh{} List 함수가 꼭 필요하지는 않음  df1[\PYGZsq{}c1\PYGZsq{}] =\PYGZgt{} [11,12,13,14,15]}
    \PYG{n}{r\PYGZus{}list}\PYG{o}{.}\PYG{n}{append}\PYG{p}{(}\PYG{n}{i}\PYG{o}{*}\PYG{o}{*}\PYG{l+m+mi}{2}\PYG{p}{)}
    
\PYG{n}{df1}\PYG{p}{[}\PYG{l+s+s1}{\PYGZsq{}}\PYG{l+s+s1}{c3}\PYG{l+s+s1}{\PYGZsq{}}\PYG{p}{]} \PYG{o}{=} \PYG{n}{r\PYGZus{}list} \PYG{c+c1}{\PYGZsh{} r\PYGZus{}list 갯수와 df1 갯수가 동일해야 함}

\PYG{n+nb}{print}\PYG{p}{(}\PYG{n}{df1}\PYG{p}{)}
\end{sphinxVerbatim}

\end{sphinxuseclass}\end{sphinxVerbatimInput}
\begin{sphinxVerbatimOutput}

\begin{sphinxuseclass}{cell_output}
\begin{sphinxVerbatim}[commandchars=\\\{\}]
   c1 c2   c3
0  11  a  121
1  12  b  144
2  13  c  169
3  14  d  196
4  15  e  225
\end{sphinxVerbatim}

\end{sphinxuseclass}\end{sphinxVerbatimOutput}

\end{sphinxuseclass}


\begin{sphinxuseclass}{cell}\begin{sphinxVerbatimInput}

\begin{sphinxuseclass}{cell_input}
\begin{sphinxVerbatim}[commandchars=\\\{\}]
\PYG{k+kn}{import} \PYG{n+nn}{pandas} \PYG{k}{as} \PYG{n+nn}{pd}

\PYG{n}{c1\PYGZus{}list} \PYG{o}{=} \PYG{p}{[}\PYG{l+m+mi}{11}\PYG{p}{,}\PYG{l+m+mi}{12}\PYG{p}{,}\PYG{l+m+mi}{13}\PYG{p}{,}\PYG{l+m+mi}{14}\PYG{p}{,}\PYG{l+m+mi}{15}\PYG{p}{]}
\PYG{n}{c2\PYGZus{}list} \PYG{o}{=} \PYG{p}{[}\PYG{l+s+s1}{\PYGZsq{}}\PYG{l+s+s1}{a}\PYG{l+s+s1}{\PYGZsq{}}\PYG{p}{,}\PYG{l+s+s1}{\PYGZsq{}}\PYG{l+s+s1}{b}\PYG{l+s+s1}{\PYGZsq{}}\PYG{p}{,}\PYG{l+s+s1}{\PYGZsq{}}\PYG{l+s+s1}{c}\PYG{l+s+s1}{\PYGZsq{}}\PYG{p}{,}\PYG{l+s+s1}{\PYGZsq{}}\PYG{l+s+s1}{d}\PYG{l+s+s1}{\PYGZsq{}}\PYG{p}{,}\PYG{l+s+s1}{\PYGZsq{}}\PYG{l+s+s1}{e}\PYG{l+s+s1}{\PYGZsq{}}\PYG{p}{]}
\PYG{n}{df1} \PYG{o}{=} \PYG{n}{pd}\PYG{o}{.}\PYG{n}{DataFrame}\PYG{p}{(}\PYG{p}{\PYGZob{}}\PYG{l+s+s1}{\PYGZsq{}}\PYG{l+s+s1}{c1}\PYG{l+s+s1}{\PYGZsq{}}\PYG{p}{:} \PYG{n}{c1\PYGZus{}list}\PYG{p}{,} \PYG{l+s+s1}{\PYGZsq{}}\PYG{l+s+s1}{c2}\PYG{l+s+s1}{\PYGZsq{}}\PYG{p}{:} \PYG{n}{c2\PYGZus{}list}\PYG{p}{\PYGZcb{}}\PYG{p}{)}

\PYG{n}{df1}\PYG{p}{[}\PYG{l+s+s1}{\PYGZsq{}}\PYG{l+s+s1}{c3}\PYG{l+s+s1}{\PYGZsq{}}\PYG{p}{]} \PYG{o}{=} \PYG{n}{df1}\PYG{p}{[}\PYG{l+s+s1}{\PYGZsq{}}\PYG{l+s+s1}{c1}\PYG{l+s+s1}{\PYGZsq{}}\PYG{p}{]}\PYG{o}{*}\PYG{o}{*}\PYG{l+m+mi}{2}
\PYG{n+nb}{print}\PYG{p}{(}\PYG{n}{df1}\PYG{p}{)}
\end{sphinxVerbatim}

\end{sphinxuseclass}\end{sphinxVerbatimInput}
\begin{sphinxVerbatimOutput}

\begin{sphinxuseclass}{cell_output}
\begin{sphinxVerbatim}[commandchars=\\\{\}]
   c1 c2   c3
0  11  a  121
1  12  b  144
2  13  c  169
3  14  d  196
4  15  e  225
\end{sphinxVerbatim}

\end{sphinxuseclass}\end{sphinxVerbatimOutput}

\end{sphinxuseclass}

\part{Concat 과 Merge}
\label{\detokenize{chapter2/2.2.2_Useful_Techniques:concat-merge}}\label{\detokenize{chapter2/2.2.2_Useful_Techniques::doc}}
\sphinxAtStartPar
Concat 과 Merge 는 두개 이상의 DataFrame/Series 을 키 값(매칭을 위한 값)으로 합칠 때 쓰는 메소드입니다.


\part{Concat}
\label{\detokenize{chapter2/2.2.2_Useful_Techniques:concat}}
\sphinxAtStartPar
먼저 concat 를 해 보겠습니다. concat 는 axis 라는 인수를 사용해서 위\sphinxhyphen{}아래로 합할 것인지, 좌\sphinxhyphen{}우로 합할 것인지 알려줍니다. 좌\sphinxhyphen{}우로 합치는 경우는 index(행) 를 기준으로 하고, 위\sphinxhyphen{}아래로 합치는 경우는 column(열) 을 기준으로 합니다. 다른 인수로는 join 이 있습니다. 주로 axis=1 로 병합(좌\sphinxhyphen{}우)하는 경우가 많은데요. 양쪽 데이터셋에 동시에 존재하는 index 만으로 합칠 때는 join=’inner’ 를 넣어주고, 모든 index 를 남기고 싶을 때는 join=’outer’ 를 넣어줍니다. 아래 예제에서 Series s1 과 Series s2 의 index 가 동일하므로, ‘inner’ 나 ‘outer’ 로 합쳐도 동일한 결과가 나옵니다. 두개의 데이터셋이 같은 지 체크하는 메소드는 equal 입니다. 체크한 결과 True 를 얻었습니다. 참고로 함수의 ()안에 커서를 놓고, ‘Shift+Tab’ 를 하면 활용 가능한 모든 인수와 설명이 나옵니다.

\begin{sphinxuseclass}{cell}\begin{sphinxVerbatimInput}

\begin{sphinxuseclass}{cell_input}
\begin{sphinxVerbatim}[commandchars=\\\{\}]
\PYG{k+kn}{import} \PYG{n+nn}{pandas} \PYG{k}{as} \PYG{n+nn}{pd}

\PYG{n}{s1} \PYG{o}{=} \PYG{n}{pd}\PYG{o}{.}\PYG{n}{Series}\PYG{p}{(}\PYG{p}{[}\PYG{l+m+mi}{1}\PYG{p}{,}\PYG{l+m+mi}{2}\PYG{p}{,}\PYG{l+m+mi}{3}\PYG{p}{,}\PYG{l+m+mi}{4}\PYG{p}{,}\PYG{l+m+mi}{5}\PYG{p}{]}\PYG{p}{,} \PYG{n}{name}\PYG{o}{=}\PYG{l+s+s1}{\PYGZsq{}}\PYG{l+s+s1}{s1}\PYG{l+s+s1}{\PYGZsq{}}\PYG{p}{)} \PYG{c+c1}{\PYGZsh{} 두 Series를 합친 후, 어느 Series 에서 알기위해 이름 지정}
\PYG{n}{s2} \PYG{o}{=} \PYG{n}{pd}\PYG{o}{.}\PYG{n}{Series}\PYG{p}{(}\PYG{p}{[}\PYG{l+s+s1}{\PYGZsq{}}\PYG{l+s+s1}{a}\PYG{l+s+s1}{\PYGZsq{}}\PYG{p}{,}\PYG{l+s+s1}{\PYGZsq{}}\PYG{l+s+s1}{b}\PYG{l+s+s1}{\PYGZsq{}}\PYG{p}{,}\PYG{l+s+s1}{\PYGZsq{}}\PYG{l+s+s1}{c}\PYG{l+s+s1}{\PYGZsq{}}\PYG{p}{,}\PYG{l+s+s1}{\PYGZsq{}}\PYG{l+s+s1}{d}\PYG{l+s+s1}{\PYGZsq{}}\PYG{p}{,}\PYG{l+s+s1}{\PYGZsq{}}\PYG{l+s+s1}{e}\PYG{l+s+s1}{\PYGZsq{}}\PYG{p}{]}\PYG{p}{,} \PYG{n}{name}\PYG{o}{=}\PYG{l+s+s1}{\PYGZsq{}}\PYG{l+s+s1}{s2}\PYG{l+s+s1}{\PYGZsq{}}\PYG{p}{)}

\PYG{n}{horizontal} \PYG{o}{=} \PYG{n}{pd}\PYG{o}{.}\PYG{n}{concat}\PYG{p}{(}\PYG{p}{[}\PYG{n}{s1}\PYG{p}{,} \PYG{n}{s2}\PYG{p}{]}\PYG{p}{,} \PYG{n}{axis}\PYG{o}{=}\PYG{l+m+mi}{1}\PYG{p}{)} \PYG{c+c1}{\PYGZsh{} axis=1 이면 index (행) 기준으로 합함. 즉. 좌\PYGZhy{}우로 합함}
\PYG{n+nb}{print}\PYG{p}{(}\PYG{n}{horizontal}\PYG{p}{)}

\PYG{n+nb}{print}\PYG{p}{(}\PYG{l+s+s1}{\PYGZsq{}}\PYG{l+s+se}{\PYGZbs{}n}\PYG{l+s+s1}{\PYGZsq{}}\PYG{p}{)}
\PYG{n}{vertical} \PYG{o}{=} \PYG{n}{pd}\PYG{o}{.}\PYG{n}{concat}\PYG{p}{(}\PYG{p}{[}\PYG{n}{s1}\PYG{p}{,} \PYG{n}{s2}\PYG{p}{]}\PYG{p}{,} \PYG{n}{axis}\PYG{o}{=}\PYG{l+m+mi}{0}\PYG{p}{)} \PYG{c+c1}{\PYGZsh{} axis=0 이면 column(열) 기준으로 합함. 즉 위\PYGZhy{}아래로 합함}
\PYG{n+nb}{print}\PYG{p}{(}\PYG{n}{vertical}\PYG{p}{)}

\PYG{n+nb}{print}\PYG{p}{(}\PYG{l+s+s1}{\PYGZsq{}}\PYG{l+s+se}{\PYGZbs{}n}\PYG{l+s+s1}{\PYGZsq{}}\PYG{p}{)}
\PYG{n}{vertical\PYGZus{}1} \PYG{o}{=} \PYG{n}{pd}\PYG{o}{.}\PYG{n}{concat}\PYG{p}{(}\PYG{p}{[}\PYG{n}{s1}\PYG{p}{,} \PYG{n}{s2}\PYG{p}{]}\PYG{p}{,} \PYG{n}{axis}\PYG{o}{=}\PYG{l+m+mi}{1}\PYG{p}{,} \PYG{n}{join}\PYG{o}{=}\PYG{l+s+s1}{\PYGZsq{}}\PYG{l+s+s1}{outer}\PYG{l+s+s1}{\PYGZsq{}}\PYG{p}{)} \PYG{c+c1}{\PYGZsh{} axis=1 인덱스 기준으로 합함. 양쪽 Series 에 존재하는 모든 index 는 남김}
\PYG{n}{vertical\PYGZus{}2} \PYG{o}{=} \PYG{n}{pd}\PYG{o}{.}\PYG{n}{concat}\PYG{p}{(}\PYG{p}{[}\PYG{n}{s1}\PYG{p}{,} \PYG{n}{s2}\PYG{p}{]}\PYG{p}{,} \PYG{n}{axis}\PYG{o}{=}\PYG{l+m+mi}{1}\PYG{p}{,} \PYG{n}{join}\PYG{o}{=}\PYG{l+s+s1}{\PYGZsq{}}\PYG{l+s+s1}{inner}\PYG{l+s+s1}{\PYGZsq{}}\PYG{p}{)} \PYG{c+c1}{\PYGZsh{} axis=1 인덱스 기준으로 합함. 양쪽 Series 에 동시에 존재하는 index 만 남김}
\PYG{n+nb}{print}\PYG{p}{(}\PYG{n}{vertical\PYGZus{}1}\PYG{o}{.}\PYG{n}{equals}\PYG{p}{(}\PYG{n}{vertical\PYGZus{}2}\PYG{p}{)}\PYG{p}{)} \PYG{c+c1}{\PYGZsh{} 두개의 DataFrame 이 서로 동일한지 체크. 인덱스가 동일하므로 동일 결과가 됨.}
\end{sphinxVerbatim}

\end{sphinxuseclass}\end{sphinxVerbatimInput}
\begin{sphinxVerbatimOutput}

\begin{sphinxuseclass}{cell_output}
\begin{sphinxVerbatim}[commandchars=\\\{\}]
   s1 s2
0   1  a
1   2  b
2   3  c
3   4  d
4   5  e


0    1
1    2
2    3
3    4
4    5
0    a
1    b
2    c
3    d
4    e
dtype: object


True
\end{sphinxVerbatim}

\end{sphinxuseclass}\end{sphinxVerbatimOutput}

\end{sphinxuseclass}
\sphinxAtStartPar
 concat 에서 두 Series 의 index 가 다르경우, 원하는 결과가 안 나온다는 것의 유의합니다. 아래 예제에서 index 가 서로 다른 Series 를 합쳐보겠습니다. join=’inner’ 조건에서는 동일한 index 가 없으므로 concat 후 결과가 없습니다. 단지 좌\sphinxhyphen{}우로 합치는 것이 목적이라면 기존의 index 를 제거하고 default index 인 숫자를 넣어주고 concat 하면 됩니다. 기존의 index 를 제거할 때는  reset\_index(drop=True) 를 합니다.

\begin{sphinxuseclass}{cell}\begin{sphinxVerbatimInput}

\begin{sphinxuseclass}{cell_input}
\begin{sphinxVerbatim}[commandchars=\\\{\}]
\PYG{n}{s3} \PYG{o}{=} \PYG{n}{pd}\PYG{o}{.}\PYG{n}{Series}\PYG{p}{(}\PYG{p}{[}\PYG{l+m+mi}{1}\PYG{p}{,}\PYG{l+m+mi}{2}\PYG{p}{,}\PYG{l+m+mi}{3}\PYG{p}{,}\PYG{l+m+mi}{4}\PYG{p}{,}\PYG{l+m+mi}{5}\PYG{p}{]}\PYG{p}{,} \PYG{n}{index} \PYG{o}{=} \PYG{p}{[}\PYG{l+s+s1}{\PYGZsq{}}\PYG{l+s+s1}{a}\PYG{l+s+s1}{\PYGZsq{}}\PYG{p}{,}\PYG{l+s+s1}{\PYGZsq{}}\PYG{l+s+s1}{b}\PYG{l+s+s1}{\PYGZsq{}}\PYG{p}{,}\PYG{l+s+s1}{\PYGZsq{}}\PYG{l+s+s1}{c}\PYG{l+s+s1}{\PYGZsq{}}\PYG{p}{,}\PYG{l+s+s1}{\PYGZsq{}}\PYG{l+s+s1}{d}\PYG{l+s+s1}{\PYGZsq{}}\PYG{p}{,}\PYG{l+s+s1}{\PYGZsq{}}\PYG{l+s+s1}{e}\PYG{l+s+s1}{\PYGZsq{}}\PYG{p}{]} \PYG{p}{,} \PYG{n}{name}\PYG{o}{=}\PYG{l+s+s1}{\PYGZsq{}}\PYG{l+s+s1}{s3}\PYG{l+s+s1}{\PYGZsq{}}\PYG{p}{)}
\PYG{n}{s4} \PYG{o}{=} \PYG{n}{pd}\PYG{o}{.}\PYG{n}{Series}\PYG{p}{(}\PYG{p}{[}\PYG{l+m+mi}{11}\PYG{p}{,}\PYG{l+m+mi}{12}\PYG{p}{,}\PYG{l+m+mi}{13}\PYG{p}{,}\PYG{l+m+mi}{14}\PYG{p}{,}\PYG{l+m+mi}{15}\PYG{p}{]}\PYG{p}{,} \PYG{n}{index} \PYG{o}{=} \PYG{p}{[}\PYG{l+s+s1}{\PYGZsq{}}\PYG{l+s+s1}{f}\PYG{l+s+s1}{\PYGZsq{}}\PYG{p}{,}\PYG{l+s+s1}{\PYGZsq{}}\PYG{l+s+s1}{g}\PYG{l+s+s1}{\PYGZsq{}}\PYG{p}{,}\PYG{l+s+s1}{\PYGZsq{}}\PYG{l+s+s1}{h}\PYG{l+s+s1}{\PYGZsq{}}\PYG{p}{,}\PYG{l+s+s1}{\PYGZsq{}}\PYG{l+s+s1}{i}\PYG{l+s+s1}{\PYGZsq{}}\PYG{p}{,}\PYG{l+s+s1}{\PYGZsq{}}\PYG{l+s+s1}{j}\PYG{l+s+s1}{\PYGZsq{}}\PYG{p}{]}\PYG{p}{,} \PYG{n}{name}\PYG{o}{=}\PYG{l+s+s1}{\PYGZsq{}}\PYG{l+s+s1}{s4}\PYG{l+s+s1}{\PYGZsq{}}\PYG{p}{)}

\PYG{n+nb}{print}\PYG{p}{(}\PYG{n}{pd}\PYG{o}{.}\PYG{n}{concat}\PYG{p}{(}\PYG{p}{[}\PYG{n}{s3}\PYG{p}{,} \PYG{n}{s4}\PYG{p}{]}\PYG{p}{,} \PYG{n}{axis}\PYG{o}{=}\PYG{l+m+mi}{1}\PYG{p}{,} \PYG{n}{join}\PYG{o}{=}\PYG{l+s+s1}{\PYGZsq{}}\PYG{l+s+s1}{inner}\PYG{l+s+s1}{\PYGZsq{}}\PYG{p}{)}\PYG{p}{)} \PYG{c+c1}{\PYGZsh{} axis=1 이면 인덱스 기준으로 합함. 즉. 좌\PYGZhy{}우로 합함}

\PYG{n+nb}{print}\PYG{p}{(}\PYG{l+s+s1}{\PYGZsq{}}\PYG{l+s+se}{\PYGZbs{}n}\PYG{l+s+s1}{\PYGZsq{}}\PYG{p}{)}
\PYG{n+nb}{print}\PYG{p}{(}\PYG{n}{pd}\PYG{o}{.}\PYG{n}{concat}\PYG{p}{(}\PYG{p}{[}\PYG{n}{s3}\PYG{o}{.}\PYG{n}{reset\PYGZus{}index}\PYG{p}{(}\PYG{n}{drop}\PYG{o}{=}\PYG{k+kc}{True}\PYG{p}{)}\PYG{p}{,} \PYG{n}{s4}\PYG{o}{.}\PYG{n}{reset\PYGZus{}index}\PYG{p}{(}\PYG{n}{drop}\PYG{o}{=}\PYG{k+kc}{True}\PYG{p}{)}\PYG{p}{]}\PYG{p}{,} \PYG{n}{axis}\PYG{o}{=}\PYG{l+m+mi}{1}\PYG{p}{,} \PYG{n}{join}\PYG{o}{=}\PYG{l+s+s1}{\PYGZsq{}}\PYG{l+s+s1}{inner}\PYG{l+s+s1}{\PYGZsq{}}\PYG{p}{)}\PYG{p}{)} \PYG{c+c1}{\PYGZsh{} axis=1 이면 인덱스 기준으로 합함. 즉. 좌\PYGZhy{}우로 합함}
\end{sphinxVerbatim}

\end{sphinxuseclass}\end{sphinxVerbatimInput}
\begin{sphinxVerbatimOutput}

\begin{sphinxuseclass}{cell_output}
\begin{sphinxVerbatim}[commandchars=\\\{\}]
Empty DataFrame
Columns: [s3, s4]
Index: []


   s3  s4
0   1  11
1   2  12
2   3  13
3   4  14
4   5  15
\end{sphinxVerbatim}

\end{sphinxuseclass}\end{sphinxVerbatimOutput}

\end{sphinxuseclass}

\part{Merge}
\label{\detokenize{chapter2/2.2.2_Useful_Techniques:merge}}
\sphinxAtStartPar
Index 가 동일하고, 단순한 병합일 때는 concat 를 쓰지만, 서로 다른 컬럼으로 병합을 할 때는 Merge 를 씁니다.
만약 두 데이터셋이 있고, 고객번호로 서로 Merge 하려고 한다고 합시다. 그런데 한 데이터셋에는 고객번호가 cust\_id 로 되어 있고, 다른 데이터셋에는 Cust\_Number 로 되어 있으면 concat 를 활용하기 어렵습니다. 이 경우는 merge 를 쓰는 것이 편리합니다. merge 는 넣어야하는 인수가 concat 보다많아, 단순한 병합은 concat 으로 합니다. 먼저 예제 DataFrame 을 생성합니다.

\begin{sphinxuseclass}{cell}\begin{sphinxVerbatimInput}

\begin{sphinxuseclass}{cell_input}
\begin{sphinxVerbatim}[commandchars=\\\{\}]
\PYG{n}{cust\PYGZus{}list} \PYG{o}{=} \PYG{p}{[}\PYG{l+m+mi}{10}\PYG{p}{,} \PYG{l+m+mi}{11}\PYG{p}{,} \PYG{l+m+mi}{12}\PYG{p}{,} \PYG{l+m+mi}{13}\PYG{p}{,} \PYG{l+m+mi}{14}\PYG{p}{,} \PYG{l+m+mi}{15}\PYG{p}{]}
\PYG{n}{product\PYGZus{}list} \PYG{o}{=} \PYG{p}{[}\PYG{l+s+s1}{\PYGZsq{}}\PYG{l+s+s1}{a}\PYG{l+s+s1}{\PYGZsq{}}\PYG{p}{,}\PYG{l+s+s1}{\PYGZsq{}}\PYG{l+s+s1}{b}\PYG{l+s+s1}{\PYGZsq{}}\PYG{p}{,}\PYG{l+s+s1}{\PYGZsq{}}\PYG{l+s+s1}{c}\PYG{l+s+s1}{\PYGZsq{}}\PYG{p}{,}\PYG{l+s+s1}{\PYGZsq{}}\PYG{l+s+s1}{d}\PYG{l+s+s1}{\PYGZsq{}}\PYG{p}{,}\PYG{l+s+s1}{\PYGZsq{}}\PYG{l+s+s1}{e}\PYG{l+s+s1}{\PYGZsq{}}\PYG{p}{,} \PYG{l+s+s1}{\PYGZsq{}}\PYG{l+s+s1}{f}\PYG{l+s+s1}{\PYGZsq{}}\PYG{p}{]}
\PYG{n}{df1} \PYG{o}{=} \PYG{n}{pd}\PYG{o}{.}\PYG{n}{DataFrame}\PYG{p}{(}\PYG{p}{\PYGZob{}}\PYG{l+s+s1}{\PYGZsq{}}\PYG{l+s+s1}{cust\PYGZus{}id}\PYG{l+s+s1}{\PYGZsq{}}\PYG{p}{:} \PYG{n}{cust\PYGZus{}list}\PYG{p}{,} \PYG{l+s+s1}{\PYGZsq{}}\PYG{l+s+s1}{product}\PYG{l+s+s1}{\PYGZsq{}}\PYG{p}{:} \PYG{n}{product\PYGZus{}list}\PYG{p}{\PYGZcb{}}\PYG{p}{)}

\PYG{n}{cust\PYGZus{}list} \PYG{o}{=} \PYG{p}{[}\PYG{l+m+mi}{12}\PYG{p}{,} \PYG{l+m+mi}{13}\PYG{p}{,} \PYG{l+m+mi}{14}\PYG{p}{,} \PYG{l+m+mi}{15}\PYG{p}{,} \PYG{l+m+mi}{16}\PYG{p}{,} \PYG{l+m+mi}{17}\PYG{p}{]}
\PYG{n}{grade\PYGZus{}list} \PYG{o}{=} \PYG{p}{[}\PYG{l+s+s1}{\PYGZsq{}}\PYG{l+s+s1}{p1}\PYG{l+s+s1}{\PYGZsq{}}\PYG{p}{,}\PYG{l+s+s1}{\PYGZsq{}}\PYG{l+s+s1}{p2}\PYG{l+s+s1}{\PYGZsq{}}\PYG{p}{,}\PYG{l+s+s1}{\PYGZsq{}}\PYG{l+s+s1}{p3}\PYG{l+s+s1}{\PYGZsq{}}\PYG{p}{,}\PYG{l+s+s1}{\PYGZsq{}}\PYG{l+s+s1}{p4}\PYG{l+s+s1}{\PYGZsq{}}\PYG{p}{,}\PYG{l+s+s1}{\PYGZsq{}}\PYG{l+s+s1}{p5}\PYG{l+s+s1}{\PYGZsq{}}\PYG{p}{,}\PYG{l+s+s1}{\PYGZsq{}}\PYG{l+s+s1}{p6}\PYG{l+s+s1}{\PYGZsq{}}\PYG{p}{]}
\PYG{n}{df2} \PYG{o}{=} \PYG{n}{pd}\PYG{o}{.}\PYG{n}{DataFrame}\PYG{p}{(}\PYG{p}{\PYGZob{}}\PYG{l+s+s1}{\PYGZsq{}}\PYG{l+s+s1}{cust\PYGZus{}number}\PYG{l+s+s1}{\PYGZsq{}}\PYG{p}{:} \PYG{n}{cust\PYGZus{}list}\PYG{p}{,} \PYG{l+s+s1}{\PYGZsq{}}\PYG{l+s+s1}{grade}\PYG{l+s+s1}{\PYGZsq{}}\PYG{p}{:} \PYG{n}{grade\PYGZus{}list}\PYG{p}{\PYGZcb{}}\PYG{p}{)}

\PYG{n+nb}{print}\PYG{p}{(}\PYG{n}{df1}\PYG{p}{)}
\PYG{n+nb}{print}\PYG{p}{(}\PYG{l+s+s1}{\PYGZsq{}}\PYG{l+s+se}{\PYGZbs{}n}\PYG{l+s+s1}{\PYGZsq{}}\PYG{p}{)}
\PYG{n+nb}{print}\PYG{p}{(}\PYG{n}{df2}\PYG{p}{)}
\end{sphinxVerbatim}

\end{sphinxuseclass}\end{sphinxVerbatimInput}
\begin{sphinxVerbatimOutput}

\begin{sphinxuseclass}{cell_output}
\begin{sphinxVerbatim}[commandchars=\\\{\}]
   cust\PYGZus{}id product
0       10       a
1       11       b
2       12       c
3       13       d
4       14       e
5       15       f


   cust\PYGZus{}number grade
0           12    p1
1           13    p2
2           14    p3
3           15    p4
4           16    p5
5           17    p6
\end{sphinxVerbatim}

\end{sphinxuseclass}\end{sphinxVerbatimOutput}

\end{sphinxuseclass}


\begin{sphinxuseclass}{cell}\begin{sphinxVerbatimInput}

\begin{sphinxuseclass}{cell_input}
\begin{sphinxVerbatim}[commandchars=\\\{\}]
\PYG{n+nb}{print}\PYG{p}{(}\PYG{n}{df1}\PYG{o}{.}\PYG{n}{set\PYGZus{}index}\PYG{p}{(}\PYG{l+s+s1}{\PYGZsq{}}\PYG{l+s+s1}{cust\PYGZus{}id}\PYG{l+s+s1}{\PYGZsq{}}\PYG{p}{)}\PYG{p}{)}
\PYG{n+nb}{print}\PYG{p}{(}\PYG{n}{df2}\PYG{o}{.}\PYG{n}{set\PYGZus{}index}\PYG{p}{(}\PYG{l+s+s1}{\PYGZsq{}}\PYG{l+s+s1}{cust\PYGZus{}number}\PYG{l+s+s1}{\PYGZsq{}}\PYG{p}{)}\PYG{p}{)}
\end{sphinxVerbatim}

\end{sphinxuseclass}\end{sphinxVerbatimInput}
\begin{sphinxVerbatimOutput}

\begin{sphinxuseclass}{cell_output}
\begin{sphinxVerbatim}[commandchars=\\\{\}]
        product
cust\PYGZus{}id        
10            a
11            b
12            c
13            d
14            e
15            f
            grade
cust\PYGZus{}number      
12             p1
13             p2
14             p3
15             p4
16             p5
17             p6
\end{sphinxVerbatim}

\end{sphinxuseclass}\end{sphinxVerbatimOutput}

\end{sphinxuseclass}


\begin{sphinxuseclass}{cell}\begin{sphinxVerbatimInput}

\begin{sphinxuseclass}{cell_input}
\begin{sphinxVerbatim}[commandchars=\\\{\}]
\PYG{n}{df1}\PYG{o}{.}\PYG{n}{set\PYGZus{}index}\PYG{p}{(}\PYG{l+s+s1}{\PYGZsq{}}\PYG{l+s+s1}{cust\PYGZus{}id}\PYG{l+s+s1}{\PYGZsq{}}\PYG{p}{)}\PYG{o}{.}\PYG{n}{merge}\PYG{p}{(}\PYG{n}{df2}\PYG{o}{.}\PYG{n}{set\PYGZus{}index}\PYG{p}{(}\PYG{l+s+s1}{\PYGZsq{}}\PYG{l+s+s1}{cust\PYGZus{}number}\PYG{l+s+s1}{\PYGZsq{}}\PYG{p}{)}\PYG{p}{,} \PYG{n}{left\PYGZus{}index}\PYG{o}{=}\PYG{k+kc}{True}\PYG{p}{,} \PYG{n}{right\PYGZus{}index}\PYG{o}{=}\PYG{k+kc}{True}\PYG{p}{,} \PYG{n}{how}\PYG{o}{=}\PYG{l+s+s1}{\PYGZsq{}}\PYG{l+s+s1}{inner}\PYG{l+s+s1}{\PYGZsq{}}\PYG{p}{)}
\end{sphinxVerbatim}

\end{sphinxuseclass}\end{sphinxVerbatimInput}
\begin{sphinxVerbatimOutput}

\begin{sphinxuseclass}{cell_output}
\begin{sphinxVerbatim}[commandchars=\\\{\}]
   product grade
12       c    p1
13       d    p2
14       e    p3
15       f    p4
\end{sphinxVerbatim}

\end{sphinxuseclass}\end{sphinxVerbatimOutput}

\end{sphinxuseclass}
\begin{sphinxuseclass}{cell}\begin{sphinxVerbatimInput}

\begin{sphinxuseclass}{cell_input}
\begin{sphinxVerbatim}[commandchars=\\\{\}]
\PYG{n}{pd}\PYG{o}{.}\PYG{n}{merge}\PYG{p}{(}\PYG{n}{left}\PYG{o}{=}\PYG{n}{df1}\PYG{o}{.}\PYG{n}{set\PYGZus{}index}\PYG{p}{(}\PYG{l+s+s1}{\PYGZsq{}}\PYG{l+s+s1}{cust\PYGZus{}id}\PYG{l+s+s1}{\PYGZsq{}}\PYG{p}{)}\PYG{p}{,} \PYG{n}{right}\PYG{o}{=}\PYG{n}{df2}\PYG{o}{.}\PYG{n}{set\PYGZus{}index}\PYG{p}{(}\PYG{l+s+s1}{\PYGZsq{}}\PYG{l+s+s1}{cust\PYGZus{}number}\PYG{l+s+s1}{\PYGZsq{}}\PYG{p}{)}\PYG{p}{,} \PYG{n}{left\PYGZus{}index}\PYG{o}{=}\PYG{k+kc}{True}\PYG{p}{,} \PYG{n}{right\PYGZus{}index}\PYG{o}{=}\PYG{k+kc}{True}\PYG{p}{,} \PYG{n}{how}\PYG{o}{=}\PYG{l+s+s1}{\PYGZsq{}}\PYG{l+s+s1}{inner}\PYG{l+s+s1}{\PYGZsq{}}\PYG{p}{)}
\end{sphinxVerbatim}

\end{sphinxuseclass}\end{sphinxVerbatimInput}
\begin{sphinxVerbatimOutput}

\begin{sphinxuseclass}{cell_output}
\begin{sphinxVerbatim}[commandchars=\\\{\}]
   product grade
12       c    p1
13       d    p2
14       e    p3
15       f    p4
\end{sphinxVerbatim}

\end{sphinxuseclass}\end{sphinxVerbatimOutput}

\end{sphinxuseclass}


\begin{sphinxuseclass}{cell}\begin{sphinxVerbatimInput}

\begin{sphinxuseclass}{cell_input}
\begin{sphinxVerbatim}[commandchars=\\\{\}]
\PYG{n}{df1}\PYG{o}{.}\PYG{n}{set\PYGZus{}index}\PYG{p}{(}\PYG{l+s+s1}{\PYGZsq{}}\PYG{l+s+s1}{cust\PYGZus{}id}\PYG{l+s+s1}{\PYGZsq{}}\PYG{p}{)}\PYG{o}{.}\PYG{n}{merge}\PYG{p}{(}\PYG{n}{df2}\PYG{o}{.}\PYG{n}{set\PYGZus{}index}\PYG{p}{(}\PYG{l+s+s1}{\PYGZsq{}}\PYG{l+s+s1}{cust\PYGZus{}number}\PYG{l+s+s1}{\PYGZsq{}}\PYG{p}{)}\PYG{p}{,} \PYG{n}{left\PYGZus{}index}\PYG{o}{=}\PYG{k+kc}{True}\PYG{p}{,} \PYG{n}{right\PYGZus{}index}\PYG{o}{=}\PYG{k+kc}{True}\PYG{p}{,} \PYG{n}{how}\PYG{o}{=}\PYG{l+s+s1}{\PYGZsq{}}\PYG{l+s+s1}{inner}\PYG{l+s+s1}{\PYGZsq{}}\PYG{p}{)}\PYG{o}{.}\PYG{n}{reset\PYGZus{}index}\PYG{p}{(}\PYG{p}{)}\PYG{o}{.}\PYG{n}{rename}\PYG{p}{(}\PYG{n}{columns}\PYG{o}{=}\PYG{p}{\PYGZob{}}\PYG{l+s+s1}{\PYGZsq{}}\PYG{l+s+s1}{index}\PYG{l+s+s1}{\PYGZsq{}}\PYG{p}{:}\PYG{l+s+s1}{\PYGZsq{}}\PYG{l+s+s1}{cust\PYGZus{}id}\PYG{l+s+s1}{\PYGZsq{}}\PYG{p}{\PYGZcb{}}\PYG{p}{)}
\end{sphinxVerbatim}

\end{sphinxuseclass}\end{sphinxVerbatimInput}
\begin{sphinxVerbatimOutput}

\begin{sphinxuseclass}{cell_output}
\begin{sphinxVerbatim}[commandchars=\\\{\}]
   cust\PYGZus{}id product grade
0       12       c    p1
1       13       d    p2
2       14       e    p3
3       15       f    p4
\end{sphinxVerbatim}

\end{sphinxuseclass}\end{sphinxVerbatimOutput}

\end{sphinxuseclass}

\part{Groupby}
\label{\detokenize{chapter2/2.2.3_Useful_Techniques:groupby}}\label{\detokenize{chapter2/2.2.3_Useful_Techniques::doc}}
\sphinxAtStartPar
Groupby 는 데이터를 요약할 때 많이 활용하는 기법입니다. 아래 예제에서 만들어진 DataFrame \sphinxhyphen{} df 의 ‘grp’ 컬럼을 이용하여 ‘a’, ‘b’, ‘c’ 등의 3 개의 그룹으로 나눌 수 있습니다.
먼저, 그룹을 무시하고 v1, v2 의 평균값을 알아봅니다. 그 다음, 그룹 별로 v1 과 v2 의 평균값을 알아봅니다.

\begin{sphinxuseclass}{cell}\begin{sphinxVerbatimInput}

\begin{sphinxuseclass}{cell_input}
\begin{sphinxVerbatim}[commandchars=\\\{\}]
\PYG{k+kn}{import} \PYG{n+nn}{pandas} \PYG{k}{as} \PYG{n+nn}{pd}

\PYG{n}{g\PYGZus{}list} \PYG{o}{=} \PYG{p}{[}\PYG{l+s+s1}{\PYGZsq{}}\PYG{l+s+s1}{a}\PYG{l+s+s1}{\PYGZsq{}}\PYG{p}{,}\PYG{l+s+s1}{\PYGZsq{}}\PYG{l+s+s1}{a}\PYG{l+s+s1}{\PYGZsq{}}\PYG{p}{,}\PYG{l+s+s1}{\PYGZsq{}}\PYG{l+s+s1}{a}\PYG{l+s+s1}{\PYGZsq{}}\PYG{p}{,}\PYG{l+s+s1}{\PYGZsq{}}\PYG{l+s+s1}{b}\PYG{l+s+s1}{\PYGZsq{}}\PYG{p}{,}\PYG{l+s+s1}{\PYGZsq{}}\PYG{l+s+s1}{b}\PYG{l+s+s1}{\PYGZsq{}}\PYG{p}{,}\PYG{l+s+s1}{\PYGZsq{}}\PYG{l+s+s1}{b}\PYG{l+s+s1}{\PYGZsq{}}\PYG{p}{,}\PYG{l+s+s1}{\PYGZsq{}}\PYG{l+s+s1}{c}\PYG{l+s+s1}{\PYGZsq{}}\PYG{p}{,}\PYG{l+s+s1}{\PYGZsq{}}\PYG{l+s+s1}{c}\PYG{l+s+s1}{\PYGZsq{}}\PYG{p}{,}\PYG{l+s+s1}{\PYGZsq{}}\PYG{l+s+s1}{c}\PYG{l+s+s1}{\PYGZsq{}}\PYG{p}{,}\PYG{l+s+s1}{\PYGZsq{}}\PYG{l+s+s1}{c}\PYG{l+s+s1}{\PYGZsq{}}\PYG{p}{]}
\PYG{n}{v1\PYGZus{}list} \PYG{o}{=} \PYG{p}{[}\PYG{l+m+mi}{1}\PYG{p}{,} \PYG{l+m+mi}{2}\PYG{p}{,} \PYG{l+m+mi}{3}\PYG{p}{,} \PYG{l+m+mi}{4}\PYG{p}{,} \PYG{l+m+mi}{5}\PYG{p}{,} \PYG{l+m+mi}{6}\PYG{p}{,} \PYG{l+m+mi}{7}\PYG{p}{,} \PYG{l+m+mi}{8}\PYG{p}{,} \PYG{l+m+mi}{9}\PYG{p}{,} \PYG{l+m+mi}{10}\PYG{p}{]}
\PYG{n}{v2\PYGZus{}list} \PYG{o}{=} \PYG{p}{[}\PYG{l+m+mi}{11}\PYG{p}{,} \PYG{l+m+mi}{12}\PYG{p}{,} \PYG{l+m+mi}{13}\PYG{p}{,} \PYG{l+m+mi}{14}\PYG{p}{,} \PYG{l+m+mi}{15}\PYG{p}{,} \PYG{l+m+mi}{16}\PYG{p}{,} \PYG{l+m+mi}{17}\PYG{p}{,} \PYG{l+m+mi}{18}\PYG{p}{,} \PYG{l+m+mi}{19}\PYG{p}{,} \PYG{l+m+mi}{20}\PYG{p}{]}

\PYG{n}{df} \PYG{o}{=}  \PYG{n}{pd}\PYG{o}{.}\PYG{n}{DataFrame}\PYG{p}{(}\PYG{p}{\PYGZob{}}\PYG{l+s+s1}{\PYGZsq{}}\PYG{l+s+s1}{grp}\PYG{l+s+s1}{\PYGZsq{}}\PYG{p}{:} \PYG{n}{g\PYGZus{}list}\PYG{p}{,} \PYG{l+s+s1}{\PYGZsq{}}\PYG{l+s+s1}{v1}\PYG{l+s+s1}{\PYGZsq{}}\PYG{p}{:} \PYG{n}{v1\PYGZus{}list}\PYG{p}{,} \PYG{l+s+s1}{\PYGZsq{}}\PYG{l+s+s1}{v2}\PYG{l+s+s1}{\PYGZsq{}}\PYG{p}{:} \PYG{n}{v2\PYGZus{}list}\PYG{p}{\PYGZcb{}}\PYG{p}{)} \PYG{c+c1}{\PYGZsh{} 그룹핑을 할 수 있는 컬럼을 가진 DataFrame 생성}
\end{sphinxVerbatim}

\end{sphinxuseclass}\end{sphinxVerbatimInput}

\end{sphinxuseclass}
\begin{sphinxuseclass}{cell}\begin{sphinxVerbatimInput}

\begin{sphinxuseclass}{cell_input}
\begin{sphinxVerbatim}[commandchars=\\\{\}]
\PYG{n}{df}\PYG{p}{[}\PYG{p}{[}\PYG{l+s+s1}{\PYGZsq{}}\PYG{l+s+s1}{v1}\PYG{l+s+s1}{\PYGZsq{}}\PYG{p}{,} \PYG{l+s+s1}{\PYGZsq{}}\PYG{l+s+s1}{v2}\PYG{l+s+s1}{\PYGZsq{}}\PYG{p}{]}\PYG{p}{]}\PYG{o}{.}\PYG{n}{mean}\PYG{p}{(}\PYG{p}{)} \PYG{c+c1}{\PYGZsh{} 전체 평균}
\end{sphinxVerbatim}

\end{sphinxuseclass}\end{sphinxVerbatimInput}
\begin{sphinxVerbatimOutput}

\begin{sphinxuseclass}{cell_output}
\begin{sphinxVerbatim}[commandchars=\\\{\}]
v1     5.5
v2    15.5
dtype: float64
\end{sphinxVerbatim}

\end{sphinxuseclass}\end{sphinxVerbatimOutput}

\end{sphinxuseclass}
\begin{sphinxuseclass}{cell}\begin{sphinxVerbatimInput}

\begin{sphinxuseclass}{cell_input}
\begin{sphinxVerbatim}[commandchars=\\\{\}]
\PYG{n}{df}\PYG{o}{.}\PYG{n}{groupby}\PYG{p}{(}\PYG{l+s+s1}{\PYGZsq{}}\PYG{l+s+s1}{grp}\PYG{l+s+s1}{\PYGZsq{}}\PYG{p}{)}\PYG{p}{[}\PYG{l+s+s1}{\PYGZsq{}}\PYG{l+s+s1}{v1}\PYG{l+s+s1}{\PYGZsq{}}\PYG{p}{]}\PYG{o}{.}\PYG{n}{mean}\PYG{p}{(}\PYG{p}{)} \PYG{c+c1}{\PYGZsh{} 그룹별 평균}
\end{sphinxVerbatim}

\end{sphinxuseclass}\end{sphinxVerbatimInput}
\begin{sphinxVerbatimOutput}

\begin{sphinxuseclass}{cell_output}
\begin{sphinxVerbatim}[commandchars=\\\{\}]
grp
a    2.0
b    5.0
c    8.5
Name: v1, dtype: float64
\end{sphinxVerbatim}

\end{sphinxuseclass}\end{sphinxVerbatimOutput}

\end{sphinxuseclass}


\begin{sphinxuseclass}{cell}\begin{sphinxVerbatimInput}

\begin{sphinxuseclass}{cell_input}
\begin{sphinxVerbatim}[commandchars=\\\{\}]
\PYG{n}{df}\PYG{o}{.}\PYG{n}{groupby}\PYG{p}{(}\PYG{l+s+s1}{\PYGZsq{}}\PYG{l+s+s1}{grp}\PYG{l+s+s1}{\PYGZsq{}}\PYG{p}{)}\PYG{p}{[}\PYG{l+s+s1}{\PYGZsq{}}\PYG{l+s+s1}{v1}\PYG{l+s+s1}{\PYGZsq{}}\PYG{p}{]}\PYG{o}{.}\PYG{n}{agg}\PYG{p}{(}\PYG{p}{[}\PYG{l+s+s1}{\PYGZsq{}}\PYG{l+s+s1}{mean}\PYG{l+s+s1}{\PYGZsq{}}\PYG{p}{,}\PYG{l+s+s1}{\PYGZsq{}}\PYG{l+s+s1}{max}\PYG{l+s+s1}{\PYGZsq{}}\PYG{p}{,}\PYG{l+s+s1}{\PYGZsq{}}\PYG{l+s+s1}{sum}\PYG{l+s+s1}{\PYGZsq{}}\PYG{p}{]}\PYG{p}{)}
\end{sphinxVerbatim}

\end{sphinxuseclass}\end{sphinxVerbatimInput}
\begin{sphinxVerbatimOutput}

\begin{sphinxuseclass}{cell_output}
\begin{sphinxVerbatim}[commandchars=\\\{\}]
     mean  max  sum
grp                
a     2.0    3    6
b     5.0    6   15
c     8.5   10   34
\end{sphinxVerbatim}

\end{sphinxuseclass}\end{sphinxVerbatimOutput}

\end{sphinxuseclass}


\begin{sphinxuseclass}{cell}\begin{sphinxVerbatimInput}

\begin{sphinxuseclass}{cell_input}
\begin{sphinxVerbatim}[commandchars=\\\{\}]
\PYG{n}{df}\PYG{o}{.}\PYG{n}{groupby}\PYG{p}{(}\PYG{l+s+s1}{\PYGZsq{}}\PYG{l+s+s1}{grp}\PYG{l+s+s1}{\PYGZsq{}}\PYG{p}{)}\PYG{p}{[}\PYG{p}{[}\PYG{l+s+s1}{\PYGZsq{}}\PYG{l+s+s1}{v1}\PYG{l+s+s1}{\PYGZsq{}}\PYG{p}{,}\PYG{l+s+s1}{\PYGZsq{}}\PYG{l+s+s1}{v2}\PYG{l+s+s1}{\PYGZsq{}}\PYG{p}{]}\PYG{p}{]}\PYG{o}{.}\PYG{n}{agg}\PYG{p}{(}\PYG{p}{[}\PYG{l+s+s1}{\PYGZsq{}}\PYG{l+s+s1}{mean}\PYG{l+s+s1}{\PYGZsq{}}\PYG{p}{,}\PYG{l+s+s1}{\PYGZsq{}}\PYG{l+s+s1}{max}\PYG{l+s+s1}{\PYGZsq{}}\PYG{p}{,}\PYG{l+s+s1}{\PYGZsq{}}\PYG{l+s+s1}{sum}\PYG{l+s+s1}{\PYGZsq{}}\PYG{p}{]}\PYG{p}{)}
\end{sphinxVerbatim}

\end{sphinxuseclass}\end{sphinxVerbatimInput}
\begin{sphinxVerbatimOutput}

\begin{sphinxuseclass}{cell_output}
\begin{sphinxVerbatim}[commandchars=\\\{\}]
      v1            v2        
    mean max sum  mean max sum
grp                           
a    2.0   3   6  12.0  13  36
b    5.0   6  15  15.0  16  45
c    8.5  10  34  18.5  20  74
\end{sphinxVerbatim}

\end{sphinxuseclass}\end{sphinxVerbatimOutput}

\end{sphinxuseclass}


\begin{sphinxuseclass}{cell}\begin{sphinxVerbatimInput}

\begin{sphinxuseclass}{cell_input}
\begin{sphinxVerbatim}[commandchars=\\\{\}]
\PYG{n}{s} \PYG{o}{=} \PYG{p}{\PYGZob{}}\PYG{l+s+s1}{\PYGZsq{}}\PYG{l+s+s1}{v1}\PYG{l+s+s1}{\PYGZsq{}}\PYG{p}{:}\PYG{l+s+s1}{\PYGZsq{}}\PYG{l+s+s1}{mean}\PYG{l+s+s1}{\PYGZsq{}}\PYG{p}{,} \PYG{l+s+s1}{\PYGZsq{}}\PYG{l+s+s1}{v2}\PYG{l+s+s1}{\PYGZsq{}}\PYG{p}{:}\PYG{l+s+s1}{\PYGZsq{}}\PYG{l+s+s1}{sum}\PYG{l+s+s1}{\PYGZsq{}}\PYG{p}{\PYGZcb{}}
\PYG{n}{df}\PYG{o}{.}\PYG{n}{groupby}\PYG{p}{(}\PYG{l+s+s1}{\PYGZsq{}}\PYG{l+s+s1}{grp}\PYG{l+s+s1}{\PYGZsq{}}\PYG{p}{)}\PYG{o}{.}\PYG{n}{agg}\PYG{p}{(}\PYG{n}{s}\PYG{p}{)}
\end{sphinxVerbatim}

\end{sphinxuseclass}\end{sphinxVerbatimInput}
\begin{sphinxVerbatimOutput}

\begin{sphinxuseclass}{cell_output}
\begin{sphinxVerbatim}[commandchars=\\\{\}]
      v1  v2
grp         
a    2.0  36
b    5.0  45
c    8.5  74
\end{sphinxVerbatim}

\end{sphinxuseclass}\end{sphinxVerbatimOutput}

\end{sphinxuseclass}


\begin{sphinxuseclass}{cell}\begin{sphinxVerbatimInput}

\begin{sphinxuseclass}{cell_input}
\begin{sphinxVerbatim}[commandchars=\\\{\}]
\PYG{n}{df}\PYG{o}{.}\PYG{n}{groupby}\PYG{p}{(}\PYG{l+s+s1}{\PYGZsq{}}\PYG{l+s+s1}{grp}\PYG{l+s+s1}{\PYGZsq{}}\PYG{p}{)}\PYG{p}{[}\PYG{l+s+s1}{\PYGZsq{}}\PYG{l+s+s1}{v1}\PYG{l+s+s1}{\PYGZsq{}}\PYG{p}{]}\PYG{o}{.}\PYG{n}{apply}\PYG{p}{(}\PYG{k}{lambda} \PYG{n}{x}\PYG{p}{:} \PYG{n}{x}\PYG{o}{.}\PYG{n}{max}\PYG{p}{(}\PYG{p}{)} \PYG{o}{\PYGZhy{}} \PYG{n}{x}\PYG{o}{.}\PYG{n}{mean}\PYG{p}{(}\PYG{p}{)}\PYG{p}{)}
\end{sphinxVerbatim}

\end{sphinxuseclass}\end{sphinxVerbatimInput}
\begin{sphinxVerbatimOutput}

\begin{sphinxuseclass}{cell_output}
\begin{sphinxVerbatim}[commandchars=\\\{\}]
grp
a    1.0
b    1.0
c    1.5
Name: v1, dtype: float64
\end{sphinxVerbatim}

\end{sphinxuseclass}\end{sphinxVerbatimOutput}

\end{sphinxuseclass}
\begin{sphinxuseclass}{cell}\begin{sphinxVerbatimInput}

\begin{sphinxuseclass}{cell_input}
\begin{sphinxVerbatim}[commandchars=\\\{\}]
\PYG{n}{df}\PYG{o}{.}\PYG{n}{groupby}\PYG{p}{(}\PYG{l+s+s1}{\PYGZsq{}}\PYG{l+s+s1}{grp}\PYG{l+s+s1}{\PYGZsq{}}\PYG{p}{)}\PYG{p}{[}\PYG{l+s+s1}{\PYGZsq{}}\PYG{l+s+s1}{v1}\PYG{l+s+s1}{\PYGZsq{}}\PYG{p}{]}\PYG{o}{.}\PYG{n}{transform}\PYG{p}{(}\PYG{k}{lambda} \PYG{n}{x}\PYG{p}{:} \PYG{n}{x}\PYG{o}{.}\PYG{n}{max}\PYG{p}{(}\PYG{p}{)} \PYG{o}{\PYGZhy{}} \PYG{n}{x}\PYG{o}{.}\PYG{n}{mean}\PYG{p}{(}\PYG{p}{)}\PYG{p}{)}
\end{sphinxVerbatim}

\end{sphinxuseclass}\end{sphinxVerbatimInput}
\begin{sphinxVerbatimOutput}

\begin{sphinxuseclass}{cell_output}
\begin{sphinxVerbatim}[commandchars=\\\{\}]
0    1.0
1    1.0
2    1.0
3    1.0
4    1.0
5    1.0
6    1.5
7    1.5
8    1.5
9    1.5
Name: v1, dtype: float64
\end{sphinxVerbatim}

\end{sphinxuseclass}\end{sphinxVerbatimOutput}

\end{sphinxuseclass}

\part{pd.cut / pd.qcut}
\label{\detokenize{chapter2/2.2.3_Useful_Techniques:br-pd-cut-pd-qcut}}
\sphinxAtStartPar
이번에는 그룹핑을 하기 위해 활용되는 pd.qcut() 혹은 pd.cut() 메소드에 대하여 알아보겠습니다. 어떤 변수를 그룹 별로 분석하고 싶습니다. 예를 들어, 섹터가 그룹이라면 groupby(‘섹터’){[}‘수익률’{]}.mean() 명령으로 섹터별로 각 섹터에 속하는 종목들의 수익율 평균울 구할 수 있습니다. 하지만 그룹 변수가 없고 연속형 변수를 구간으로 나누어 그룹화하고 싶은 경우 pd.cut() 나 pd.qcut() 을 이용합니다. pd.cut 은 직접 구간을 지정해 그룹을 만들고, pd.qcut 은 분위 수를 이용하여 구간을 만듭니다.

\begin{sphinxVerbatim}[commandchars=\\\{\}]
\PYG{n}{pd}\PYG{o}{.}\PYG{n}{qcut}\PYG{p}{(}\PYG{n}{Series}\PYG{p}{,} \PYG{n}{q}\PYG{o}{=}\PYG{l+m+mi}{10}\PYG{p}{)} \PYG{c+c1}{\PYGZsh{} 십 분위수로 구간 생성}
\PYG{n}{pd}\PYG{o}{.}\PYG{n}{cut}\PYG{p}{(}\PYG{n}{Series}\PYG{p}{,} \PYG{n}{bins}\PYG{o}{=}\PYG{p}{[}\PYG{n}{a1}\PYG{p}{,} \PYG{n}{a2}\PYG{p}{,} \PYG{n}{a3}\PYG{p}{]}\PYG{p}{)} \PYG{c+c1}{\PYGZsh{} bins 인수를 이용하면  (a1, a2], (a2, a3]로 구간 생성}
\end{sphinxVerbatim}

\sphinxAtStartPar
np.arange(100) 을 이용하여 0 부터 99까지 값을 생성한 후 pd.qcut 와 pd.cut 을 사용 해 보겠습니다.

\begin{sphinxuseclass}{cell}\begin{sphinxVerbatimInput}

\begin{sphinxuseclass}{cell_input}
\begin{sphinxVerbatim}[commandchars=\\\{\}]
\PYG{k+kn}{import} \PYG{n+nn}{numpy} \PYG{k}{as} \PYG{n+nn}{np}
\PYG{k+kn}{import} \PYG{n+nn}{pandas} \PYG{k}{as} \PYG{n+nn}{pd}

\PYG{n}{a\PYGZus{}list} \PYG{o}{=} \PYG{n}{np}\PYG{o}{.}\PYG{n}{arange}\PYG{p}{(}\PYG{l+m+mi}{100}\PYG{p}{)}
\PYG{n}{df} \PYG{o}{=} \PYG{n}{pd}\PYG{o}{.}\PYG{n}{DataFrame}\PYG{p}{(}\PYG{p}{\PYGZob{}}\PYG{l+s+s1}{\PYGZsq{}}\PYG{l+s+s1}{a}\PYG{l+s+s1}{\PYGZsq{}}\PYG{p}{:} \PYG{n}{a\PYGZus{}list}\PYG{p}{\PYGZcb{}}\PYG{p}{)} 
\PYG{n}{df}\PYG{o}{.}\PYG{n}{head}\PYG{p}{(}\PYG{p}{)}
\end{sphinxVerbatim}

\end{sphinxuseclass}\end{sphinxVerbatimInput}
\begin{sphinxVerbatimOutput}

\begin{sphinxuseclass}{cell_output}
\begin{sphinxVerbatim}[commandchars=\\\{\}]
   a
0  0
1  1
2  2
3  3
4  4
\end{sphinxVerbatim}

\end{sphinxuseclass}\end{sphinxVerbatimOutput}

\end{sphinxuseclass}
\sphinxAtStartPar
10 이하의 숫자와 90 초과의 숫자는 해댱 구간이 없어서 그룹핑이 되지 않았습니다.

\begin{sphinxuseclass}{cell}\begin{sphinxVerbatimInput}

\begin{sphinxuseclass}{cell_input}
\begin{sphinxVerbatim}[commandchars=\\\{\}]
\PYG{n}{rank} \PYG{o}{=} \PYG{n}{pd}\PYG{o}{.}\PYG{n}{cut}\PYG{p}{(}\PYG{n}{df}\PYG{p}{[}\PYG{l+s+s1}{\PYGZsq{}}\PYG{l+s+s1}{a}\PYG{l+s+s1}{\PYGZsq{}}\PYG{p}{]}\PYG{p}{,} \PYG{n}{bins}\PYG{o}{=}\PYG{p}{[}\PYG{l+m+mi}{10}\PYG{p}{,} \PYG{l+m+mi}{25}\PYG{p}{,} \PYG{l+m+mi}{75}\PYG{p}{,} \PYG{l+m+mi}{90}\PYG{p}{]}\PYG{p}{)}
\PYG{n}{df}\PYG{o}{.}\PYG{n}{groupby}\PYG{p}{(}\PYG{n}{rank}\PYG{p}{)}\PYG{p}{[}\PYG{l+s+s1}{\PYGZsq{}}\PYG{l+s+s1}{a}\PYG{l+s+s1}{\PYGZsq{}}\PYG{p}{]}\PYG{o}{.}\PYG{n}{agg}\PYG{p}{(}\PYG{p}{[}\PYG{l+s+s1}{\PYGZsq{}}\PYG{l+s+s1}{min}\PYG{l+s+s1}{\PYGZsq{}}\PYG{p}{,}\PYG{l+s+s1}{\PYGZsq{}}\PYG{l+s+s1}{max}\PYG{l+s+s1}{\PYGZsq{}}\PYG{p}{,}\PYG{l+s+s1}{\PYGZsq{}}\PYG{l+s+s1}{count}\PYG{l+s+s1}{\PYGZsq{}}\PYG{p}{]}\PYG{p}{)}
\end{sphinxVerbatim}

\end{sphinxuseclass}\end{sphinxVerbatimInput}
\begin{sphinxVerbatimOutput}

\begin{sphinxuseclass}{cell_output}
\begin{sphinxVerbatim}[commandchars=\\\{\}]
          min  max  count
a                        
(10, 25]   11   25     15
(25, 75]   26   75     50
(75, 90]   76   90     15
\end{sphinxVerbatim}

\end{sphinxuseclass}\end{sphinxVerbatimOutput}

\end{sphinxuseclass}
\sphinxAtStartPar
bins 구간이 모든 값을 포함하도록 하기 위해서는 아래와 같이 bins 를 설정합니다.

\begin{sphinxuseclass}{cell}\begin{sphinxVerbatimInput}

\begin{sphinxuseclass}{cell_input}
\begin{sphinxVerbatim}[commandchars=\\\{\}]
\PYG{n}{rank} \PYG{o}{=} \PYG{n}{pd}\PYG{o}{.}\PYG{n}{cut}\PYG{p}{(}\PYG{n}{df}\PYG{p}{[}\PYG{l+s+s1}{\PYGZsq{}}\PYG{l+s+s1}{a}\PYG{l+s+s1}{\PYGZsq{}}\PYG{p}{]}\PYG{p}{,} \PYG{n}{bins}\PYG{o}{=}\PYG{p}{[}\PYG{o}{\PYGZhy{}}\PYG{n}{np}\PYG{o}{.}\PYG{n}{inf}\PYG{p}{,} \PYG{l+m+mi}{10}\PYG{p}{,} \PYG{l+m+mi}{25}\PYG{p}{,} \PYG{l+m+mi}{75}\PYG{p}{,} \PYG{l+m+mi}{90}\PYG{p}{,} \PYG{n}{np}\PYG{o}{.}\PYG{n}{inf}\PYG{p}{]}\PYG{p}{)}
\PYG{n}{df}\PYG{o}{.}\PYG{n}{groupby}\PYG{p}{(}\PYG{n}{rank}\PYG{p}{)}\PYG{o}{.}\PYG{n}{agg}\PYG{p}{(}\PYG{p}{[}\PYG{l+s+s1}{\PYGZsq{}}\PYG{l+s+s1}{min}\PYG{l+s+s1}{\PYGZsq{}}\PYG{p}{,}\PYG{l+s+s1}{\PYGZsq{}}\PYG{l+s+s1}{max}\PYG{l+s+s1}{\PYGZsq{}}\PYG{p}{,}\PYG{l+s+s1}{\PYGZsq{}}\PYG{l+s+s1}{count}\PYG{l+s+s1}{\PYGZsq{}}\PYG{p}{]}\PYG{p}{)}
\end{sphinxVerbatim}

\end{sphinxuseclass}\end{sphinxVerbatimInput}
\begin{sphinxVerbatimOutput}

\begin{sphinxuseclass}{cell_output}
\begin{sphinxVerbatim}[commandchars=\\\{\}]
               a          
             min max count
a                         
(\PYGZhy{}inf, 10.0]   0  10    11
(10.0, 25.0]  11  25    15
(25.0, 75.0]  26  75    50
(75.0, 90.0]  76  90    15
(90.0, inf]   91  99     9
\end{sphinxVerbatim}

\end{sphinxuseclass}\end{sphinxVerbatimOutput}

\end{sphinxuseclass}
\sphinxAtStartPar
이번에는 제가 주로 활용하는 pd.qcut 입니다. pd.cut 은 bins 로 구간을 설정해야 하나,
qcut 는 q 인수로 분위수를 이용하여 구간을 만듭니다. 아래 결과를 보시면 q 의 역할을 아실 수 있을 것이라고 생각합니다.

\begin{sphinxuseclass}{cell}\begin{sphinxVerbatimInput}

\begin{sphinxuseclass}{cell_input}
\begin{sphinxVerbatim}[commandchars=\\\{\}]
\PYG{n}{rank} \PYG{o}{=} \PYG{n}{pd}\PYG{o}{.}\PYG{n}{qcut}\PYG{p}{(}\PYG{n}{df}\PYG{p}{[}\PYG{l+s+s1}{\PYGZsq{}}\PYG{l+s+s1}{a}\PYG{l+s+s1}{\PYGZsq{}}\PYG{p}{]}\PYG{p}{,} \PYG{n}{q}\PYG{o}{=}\PYG{l+m+mi}{10}\PYG{p}{)}
\PYG{n}{df}\PYG{o}{.}\PYG{n}{groupby}\PYG{p}{(}\PYG{n}{rank}\PYG{p}{)}\PYG{p}{[}\PYG{l+s+s1}{\PYGZsq{}}\PYG{l+s+s1}{a}\PYG{l+s+s1}{\PYGZsq{}}\PYG{p}{]}\PYG{o}{.}\PYG{n}{agg}\PYG{p}{(}\PYG{p}{[}\PYG{l+s+s1}{\PYGZsq{}}\PYG{l+s+s1}{min}\PYG{l+s+s1}{\PYGZsq{}}\PYG{p}{,}\PYG{l+s+s1}{\PYGZsq{}}\PYG{l+s+s1}{max}\PYG{l+s+s1}{\PYGZsq{}}\PYG{p}{,}\PYG{l+s+s1}{\PYGZsq{}}\PYG{l+s+s1}{count}\PYG{l+s+s1}{\PYGZsq{}}\PYG{p}{]}\PYG{p}{)}
\end{sphinxVerbatim}

\end{sphinxuseclass}\end{sphinxVerbatimInput}
\begin{sphinxVerbatimOutput}

\begin{sphinxuseclass}{cell_output}
\begin{sphinxVerbatim}[commandchars=\\\{\}]
               min  max  count
a                             
(\PYGZhy{}0.001, 9.9]    0    9     10
(9.9, 19.8]     10   19     10
(19.8, 29.7]    20   29     10
(29.7, 39.6]    30   39     10
(39.6, 49.5]    40   49     10
(49.5, 59.4]    50   59     10
(59.4, 69.3]    60   69     10
(69.3, 79.2]    70   79     10
(79.2, 89.1]    80   89     10
(89.1, 99.0]    90   99     10
\end{sphinxVerbatim}

\end{sphinxuseclass}\end{sphinxVerbatimOutput}

\end{sphinxuseclass}
\begin{sphinxuseclass}{cell}\begin{sphinxVerbatimInput}

\begin{sphinxuseclass}{cell_input}
\begin{sphinxVerbatim}[commandchars=\\\{\}]
\PYG{n}{rank} \PYG{o}{=} \PYG{n}{pd}\PYG{o}{.}\PYG{n}{qcut}\PYG{p}{(}\PYG{n}{df}\PYG{p}{[}\PYG{l+s+s1}{\PYGZsq{}}\PYG{l+s+s1}{a}\PYG{l+s+s1}{\PYGZsq{}}\PYG{p}{]}\PYG{p}{,} \PYG{n}{q}\PYG{o}{=}\PYG{l+m+mi}{5}\PYG{p}{)}
\PYG{n}{df}\PYG{o}{.}\PYG{n}{groupby}\PYG{p}{(}\PYG{n}{rank}\PYG{p}{)}\PYG{p}{[}\PYG{l+s+s1}{\PYGZsq{}}\PYG{l+s+s1}{a}\PYG{l+s+s1}{\PYGZsq{}}\PYG{p}{]}\PYG{o}{.}\PYG{n}{agg}\PYG{p}{(}\PYG{p}{[}\PYG{l+s+s1}{\PYGZsq{}}\PYG{l+s+s1}{min}\PYG{l+s+s1}{\PYGZsq{}}\PYG{p}{,}\PYG{l+s+s1}{\PYGZsq{}}\PYG{l+s+s1}{max}\PYG{l+s+s1}{\PYGZsq{}}\PYG{p}{,}\PYG{l+s+s1}{\PYGZsq{}}\PYG{l+s+s1}{count}\PYG{l+s+s1}{\PYGZsq{}}\PYG{p}{]}\PYG{p}{)}
\end{sphinxVerbatim}

\end{sphinxuseclass}\end{sphinxVerbatimInput}
\begin{sphinxVerbatimOutput}

\begin{sphinxuseclass}{cell_output}
\begin{sphinxVerbatim}[commandchars=\\\{\}]
                min  max  count
a                              
(\PYGZhy{}0.001, 19.8]    0   19     20
(19.8, 39.6]     20   39     20
(39.6, 59.4]     40   59     20
(59.4, 79.2]     60   79     20
(79.2, 99.0]     80   99     20
\end{sphinxVerbatim}

\end{sphinxuseclass}\end{sphinxVerbatimOutput}

\end{sphinxuseclass}

\part{Resample}
\label{\detokenize{chapter2/2.2.4_Useful_Techniques:resample}}\label{\detokenize{chapter2/2.2.4_Useful_Techniques::doc}}
\sphinxAtStartPar
Resample 은 시간데이터를 다른 시간 단위로 변경하고 싶을 때 활용합니다. 예를 들면, 초 단위 데이터를 일단위 혹은 월단위 데이터로 변경 할 수 있습니다. 연습을 위하여 시간 레벨의 데이터가 필요합니다. 시간레벨 데이터는 FinanceDataReader 패키지에서 제공하는 일봉 데이터를 활용하겠습니다. FinanceDataReader 는 이승준님이 금융자료 분석을 하시는 분들을 위하여 만들어 주신 정말 유용한 패키지입니다. 자세한 내용은 아래 링크에 설명이 되어 있습니다.
\sphinxurl{https://financedata.github.io/posts/finance-data-reader-users-guide.html} 또한, 이승준님이 Pycon 에서 엑셀에 비하여 파이썬의 장점에 대하여 강연하시는 내용이 유투브에 있습니다. \sphinxurl{https://www.youtube.com/watch?v=w7Q\_eKN5r-I}


\part{FinanceDataReader}
\label{\detokenize{chapter2/2.2.4_Useful_Techniques:financedatareader}}
\sphinxAtStartPar
FinanceDataReader 를 import 합니다. DataReader 함수에 종목코드, 시작일, 종료일을 인수로 넣어주면 아래와 같이 일봉데이터를 리턴합니다. 출력해보면 Date 가 index 로 되어 있음을 알 수 있습니다.

\begin{sphinxuseclass}{cell}\begin{sphinxVerbatimInput}

\begin{sphinxuseclass}{cell_input}
\begin{sphinxVerbatim}[commandchars=\\\{\}]
\PYG{k+kn}{import} \PYG{n+nn}{FinanceDataReader} \PYG{k}{as} \PYG{n+nn}{fdr}

\PYG{n}{code} \PYG{o}{=} \PYG{l+s+s1}{\PYGZsq{}}\PYG{l+s+s1}{005930}\PYG{l+s+s1}{\PYGZsq{}} \PYG{c+c1}{\PYGZsh{} 삼성전자}
\PYG{n}{stock\PYGZus{}data} \PYG{o}{=} \PYG{n}{fdr}\PYG{o}{.}\PYG{n}{DataReader}\PYG{p}{(}\PYG{n}{code}\PYG{p}{,} \PYG{n}{start}\PYG{o}{=}\PYG{l+s+s1}{\PYGZsq{}}\PYG{l+s+s1}{2021\PYGZhy{}01\PYGZhy{}03}\PYG{l+s+s1}{\PYGZsq{}}\PYG{p}{,} \PYG{n}{end}\PYG{o}{=}\PYG{l+s+s1}{\PYGZsq{}}\PYG{l+s+s1}{2021\PYGZhy{}12\PYGZhy{}31}\PYG{l+s+s1}{\PYGZsq{}}\PYG{p}{)} 

\PYG{n}{stock\PYGZus{}data}\PYG{o}{.}\PYG{n}{head}\PYG{p}{(}\PYG{p}{)}\PYG{o}{.}\PYG{n}{style}\PYG{o}{.}\PYG{n}{set\PYGZus{}table\PYGZus{}attributes}\PYG{p}{(}\PYG{l+s+s1}{\PYGZsq{}}\PYG{l+s+s1}{style=}\PYG{l+s+s1}{\PYGZdq{}}\PYG{l+s+s1}{font\PYGZhy{}size: 12px}\PYG{l+s+s1}{\PYGZdq{}}\PYG{l+s+s1}{\PYGZsq{}}\PYG{p}{)} \PYG{c+c1}{\PYGZsh{} head 메소드는 처음 5 row 만 출력합니다.}
\end{sphinxVerbatim}

\end{sphinxuseclass}\end{sphinxVerbatimInput}
\begin{sphinxVerbatimOutput}

\begin{sphinxuseclass}{cell_output}
\begin{sphinxVerbatim}[commandchars=\\\{\}]
\PYGZlt{}pandas.io.formats.style.Styler at 0x1eab9282310\PYGZgt{}
\end{sphinxVerbatim}

\end{sphinxuseclass}\end{sphinxVerbatimOutput}

\end{sphinxuseclass}


\begin{sphinxuseclass}{cell}\begin{sphinxVerbatimInput}

\begin{sphinxuseclass}{cell_input}
\begin{sphinxVerbatim}[commandchars=\\\{\}]
\PYG{k+kn}{import} \PYG{n+nn}{pandas} \PYG{k}{as} \PYG{n+nn}{pd}
\PYG{n}{pd}\PYG{o}{.}\PYG{n}{options}\PYG{o}{.}\PYG{n}{display}\PYG{o}{.}\PYG{n}{float\PYGZus{}format} \PYG{o}{=} \PYG{l+s+s1}{\PYGZsq{}}\PYG{l+s+si}{\PYGZob{}:,.0f\PYGZcb{}}\PYG{l+s+s1}{\PYGZsq{}}\PYG{o}{.}\PYG{n}{format}
\PYG{n}{stock\PYGZus{}data}\PYG{o}{.}\PYG{n}{resample}\PYG{p}{(}\PYG{l+s+s1}{\PYGZsq{}}\PYG{l+s+s1}{M}\PYG{l+s+s1}{\PYGZsq{}}\PYG{p}{)}\PYG{p}{[}\PYG{l+s+s1}{\PYGZsq{}}\PYG{l+s+s1}{Close}\PYG{l+s+s1}{\PYGZsq{}}\PYG{p}{]}\PYG{o}{.}\PYG{n}{agg}\PYG{p}{(}\PYG{p}{[}\PYG{l+s+s1}{\PYGZsq{}}\PYG{l+s+s1}{mean}\PYG{l+s+s1}{\PYGZsq{}}\PYG{p}{,}\PYG{l+s+s1}{\PYGZsq{}}\PYG{l+s+s1}{max}\PYG{l+s+s1}{\PYGZsq{}}\PYG{p}{,}\PYG{l+s+s1}{\PYGZsq{}}\PYG{l+s+s1}{min}\PYG{l+s+s1}{\PYGZsq{}}\PYG{p}{]}\PYG{p}{)}\PYG{o}{.}\PYG{n}{head}\PYG{p}{(}\PYG{p}{)}\PYG{o}{.}\PYG{n}{style}\PYG{o}{.}\PYG{n}{set\PYGZus{}table\PYGZus{}attributes}\PYG{p}{(}\PYG{l+s+s1}{\PYGZsq{}}\PYG{l+s+s1}{style=}\PYG{l+s+s1}{\PYGZdq{}}\PYG{l+s+s1}{font\PYGZhy{}size: 12px}\PYG{l+s+s1}{\PYGZdq{}}\PYG{l+s+s1}{\PYGZsq{}}\PYG{p}{)} \PYG{c+c1}{\PYGZsh{} 처음 5개만 출력}
\end{sphinxVerbatim}

\end{sphinxuseclass}\end{sphinxVerbatimInput}
\begin{sphinxVerbatimOutput}

\begin{sphinxuseclass}{cell_output}
\begin{sphinxVerbatim}[commandchars=\\\{\}]
\PYGZlt{}pandas.io.formats.style.Styler at 0x1eab69f9af0\PYGZgt{}
\end{sphinxVerbatim}

\end{sphinxuseclass}\end{sphinxVerbatimOutput}

\end{sphinxuseclass}
\sphinxAtStartPar
주별로 요약할 수 도 있습니다. 이번에는 resample(‘W’) 라고 해 줍니다. Resample 이 정말 유용한 기능이라는 것을 직감하셨을 것으로 생각합니다. 역시 한 주(월요일 \textasciitilde{} 일요일)의 마지막날이 Index 로 들어가 있습니다. 디폴트는 일요일입니다.

\begin{sphinxuseclass}{cell}\begin{sphinxVerbatimInput}

\begin{sphinxuseclass}{cell_input}
\begin{sphinxVerbatim}[commandchars=\\\{\}]
\PYG{n}{pd}\PYG{o}{.}\PYG{n}{options}\PYG{o}{.}\PYG{n}{display}\PYG{o}{.}\PYG{n}{float\PYGZus{}format} \PYG{o}{=} \PYG{l+s+s1}{\PYGZsq{}}\PYG{l+s+si}{\PYGZob{}:,.0f\PYGZcb{}}\PYG{l+s+s1}{\PYGZsq{}}\PYG{o}{.}\PYG{n}{format}
\PYG{n}{stock\PYGZus{}data}\PYG{o}{.}\PYG{n}{resample}\PYG{p}{(}\PYG{l+s+s1}{\PYGZsq{}}\PYG{l+s+s1}{W}\PYG{l+s+s1}{\PYGZsq{}}\PYG{p}{)}\PYG{p}{[}\PYG{l+s+s1}{\PYGZsq{}}\PYG{l+s+s1}{Close}\PYG{l+s+s1}{\PYGZsq{}}\PYG{p}{]}\PYG{o}{.}\PYG{n}{agg}\PYG{p}{(}\PYG{p}{[}\PYG{l+s+s1}{\PYGZsq{}}\PYG{l+s+s1}{mean}\PYG{l+s+s1}{\PYGZsq{}}\PYG{p}{,}\PYG{l+s+s1}{\PYGZsq{}}\PYG{l+s+s1}{max}\PYG{l+s+s1}{\PYGZsq{}}\PYG{p}{,}\PYG{l+s+s1}{\PYGZsq{}}\PYG{l+s+s1}{min}\PYG{l+s+s1}{\PYGZsq{}}\PYG{p}{]}\PYG{p}{)}\PYG{o}{.}\PYG{n}{head}\PYG{p}{(}\PYG{p}{)}\PYG{o}{.}\PYG{n}{style}\PYG{o}{.}\PYG{n}{set\PYGZus{}table\PYGZus{}attributes}\PYG{p}{(}\PYG{l+s+s1}{\PYGZsq{}}\PYG{l+s+s1}{style=}\PYG{l+s+s1}{\PYGZdq{}}\PYG{l+s+s1}{font\PYGZhy{}size: 12px}\PYG{l+s+s1}{\PYGZdq{}}\PYG{l+s+s1}{\PYGZsq{}}\PYG{p}{)}
\end{sphinxVerbatim}

\end{sphinxuseclass}\end{sphinxVerbatimInput}
\begin{sphinxVerbatimOutput}

\begin{sphinxuseclass}{cell_output}
\begin{sphinxVerbatim}[commandchars=\\\{\}]
\PYGZlt{}pandas.io.formats.style.Styler at 0x1eab947fdc0\PYGZgt{}
\end{sphinxVerbatim}

\end{sphinxuseclass}\end{sphinxVerbatimOutput}

\end{sphinxuseclass}

\part{Pickle}
\label{\detokenize{chapter2/2.2.5_Useful_Techniques:pickle}}\label{\detokenize{chapter2/2.2.5_Useful_Techniques::doc}}
\sphinxAtStartPar
Pickle 은 사전적으로 절여서 저장해 놓는다는 말인데요. 파이썬에서 데이터를 저장해 놓을 때 쓰는 패키지입니다. 파이썬 언어로 만들어진 데이터는 RAM 메모리에 존재합니다. 따라서, 컴퓨터가 꺼지면 자동으로 데이터가 사라지게 됩니다. 그래서, 저는 pickle 를 이용해서 데이터 작업 중간에 데이터를 저장합니다. 파이썬 DataFrame 의 저장은 csv, excel, json 등 다양한 형식으로 저장할 수 있으나, 파이썬의 데이터 타입을 손상시키지 않고, 원형대로 저장하고 불러올 수 있는 pickle 이 제일 편리합니다. 삼성전자 일봉데이터를 가져와서 피클로 저장해 보겠습니다.

\begin{sphinxuseclass}{cell}\begin{sphinxVerbatimInput}

\begin{sphinxuseclass}{cell_input}
\begin{sphinxVerbatim}[commandchars=\\\{\}]
\PYG{k+kn}{import} \PYG{n+nn}{FinanceDataReader} \PYG{k}{as} \PYG{n+nn}{fdr} 

\PYG{n}{code} \PYG{o}{=} \PYG{l+s+s1}{\PYGZsq{}}\PYG{l+s+s1}{005930}\PYG{l+s+s1}{\PYGZsq{}} \PYG{c+c1}{\PYGZsh{} 삼성전자}
\PYG{n}{stock\PYGZus{}data} \PYG{o}{=} \PYG{n}{fdr}\PYG{o}{.}\PYG{n}{DataReader}\PYG{p}{(}\PYG{n}{code}\PYG{p}{,} \PYG{n}{start}\PYG{o}{=}\PYG{l+s+s1}{\PYGZsq{}}\PYG{l+s+s1}{2021\PYGZhy{}01\PYGZhy{}03}\PYG{l+s+s1}{\PYGZsq{}}\PYG{p}{,} \PYG{n}{end}\PYG{o}{=}\PYG{l+s+s1}{\PYGZsq{}}\PYG{l+s+s1}{2021\PYGZhy{}12\PYGZhy{}31}\PYG{l+s+s1}{\PYGZsq{}}\PYG{p}{)} 

\PYG{n}{stock\PYGZus{}data}\PYG{o}{.}\PYG{n}{to\PYGZus{}pickle}\PYG{p}{(}\PYG{l+s+s1}{\PYGZsq{}}\PYG{l+s+s1}{stock\PYGZus{}data.pkl}\PYG{l+s+s1}{\PYGZsq{}}\PYG{p}{)} \PYG{c+c1}{\PYGZsh{} 디렉토리를 지정하지 않으면 현재 작업 폴더에 저장이 됩니다.}
\end{sphinxVerbatim}

\end{sphinxuseclass}\end{sphinxVerbatimInput}

\end{sphinxuseclass}


\begin{sphinxuseclass}{cell}\begin{sphinxVerbatimInput}

\begin{sphinxuseclass}{cell_input}
\begin{sphinxVerbatim}[commandchars=\\\{\}]
\PYG{k+kn}{import} \PYG{n+nn}{pandas} \PYG{k}{as} \PYG{n+nn}{pd}
\PYG{n}{stock\PYGZus{}data} \PYG{o}{=} \PYG{n}{pd}\PYG{o}{.}\PYG{n}{read\PYGZus{}pickle}\PYG{p}{(}\PYG{l+s+s1}{\PYGZsq{}}\PYG{l+s+s1}{stock\PYGZus{}data.pkl}\PYG{l+s+s1}{\PYGZsq{}}\PYG{p}{)}
\PYG{n}{stock\PYGZus{}data}\PYG{o}{.}\PYG{n}{head}\PYG{p}{(}\PYG{p}{)}\PYG{o}{.}\PYG{n}{style}\PYG{o}{.}\PYG{n}{set\PYGZus{}table\PYGZus{}attributes}\PYG{p}{(}\PYG{l+s+s1}{\PYGZsq{}}\PYG{l+s+s1}{style=}\PYG{l+s+s1}{\PYGZdq{}}\PYG{l+s+s1}{font\PYGZhy{}size: 12px}\PYG{l+s+s1}{\PYGZdq{}}\PYG{l+s+s1}{\PYGZsq{}}\PYG{p}{)}
\end{sphinxVerbatim}

\end{sphinxuseclass}\end{sphinxVerbatimInput}
\begin{sphinxVerbatimOutput}

\begin{sphinxuseclass}{cell_output}
\begin{sphinxVerbatim}[commandchars=\\\{\}]
\PYGZlt{}pandas.io.formats.style.Styler at 0x23a6792aa90\PYGZgt{}
\end{sphinxVerbatim}

\end{sphinxuseclass}\end{sphinxVerbatimOutput}

\end{sphinxuseclass}


\begin{sphinxuseclass}{cell}\begin{sphinxVerbatimInput}

\begin{sphinxuseclass}{cell_input}
\begin{sphinxVerbatim}[commandchars=\\\{\}]
\PYG{k+kn}{import} \PYG{n+nn}{pickle}

\PYG{k}{with} \PYG{n+nb}{open}\PYG{p}{(}\PYG{l+s+s1}{\PYGZsq{}}\PYG{l+s+s1}{stock\PYGZus{}data.pkl}\PYG{l+s+s1}{\PYGZsq{}}\PYG{p}{,} \PYG{l+s+s1}{\PYGZsq{}}\PYG{l+s+s1}{wb}\PYG{l+s+s1}{\PYGZsq{}}\PYG{p}{)} \PYG{k}{as} \PYG{n}{file}\PYG{p}{:}    \PYG{c+c1}{\PYGZsh{} Binary 파일로 저징}
    \PYG{n}{pickle}\PYG{o}{.}\PYG{n}{dump}\PYG{p}{(}\PYG{n}{stock\PYGZus{}data}\PYG{p}{,} \PYG{n}{file}\PYG{p}{)}
    
\PYG{k}{with} \PYG{n+nb}{open}\PYG{p}{(}\PYG{l+s+s1}{\PYGZsq{}}\PYG{l+s+s1}{stock\PYGZus{}data.pkl}\PYG{l+s+s1}{\PYGZsq{}}\PYG{p}{,} \PYG{l+s+s1}{\PYGZsq{}}\PYG{l+s+s1}{rb}\PYG{l+s+s1}{\PYGZsq{}}\PYG{p}{)} \PYG{k}{as} \PYG{n}{file}\PYG{p}{:}    \PYG{c+c1}{\PYGZsh{} 저장된 binary 파일 읽기}
    \PYG{n}{stock\PYGZus{}data} \PYG{o}{=} \PYG{n}{pickle}\PYG{o}{.}\PYG{n}{load}\PYG{p}{(}\PYG{n}{file}\PYG{p}{)}    
\end{sphinxVerbatim}

\end{sphinxuseclass}\end{sphinxVerbatimInput}

\end{sphinxuseclass}
\begin{sphinxuseclass}{cell}\begin{sphinxVerbatimInput}

\begin{sphinxuseclass}{cell_input}
\begin{sphinxVerbatim}[commandchars=\\\{\}]
\PYG{n}{stock\PYGZus{}data}\PYG{o}{.}\PYG{n}{head}\PYG{p}{(}\PYG{p}{)}\PYG{o}{.}\PYG{n}{style}\PYG{o}{.}\PYG{n}{set\PYGZus{}table\PYGZus{}attributes}\PYG{p}{(}\PYG{l+s+s1}{\PYGZsq{}}\PYG{l+s+s1}{style=}\PYG{l+s+s1}{\PYGZdq{}}\PYG{l+s+s1}{font\PYGZhy{}size: 12px}\PYG{l+s+s1}{\PYGZdq{}}\PYG{l+s+s1}{\PYGZsq{}}\PYG{p}{)}
\end{sphinxVerbatim}

\end{sphinxuseclass}\end{sphinxVerbatimInput}
\begin{sphinxVerbatimOutput}

\begin{sphinxuseclass}{cell_output}
\begin{sphinxVerbatim}[commandchars=\\\{\}]
\PYGZlt{}pandas.io.formats.style.Styler at 0x23a67948880\PYGZgt{}
\end{sphinxVerbatim}

\end{sphinxuseclass}\end{sphinxVerbatimOutput}

\end{sphinxuseclass}

\part{Shift}
\label{\detokenize{chapter2/2.2.6_Useful_Techniques:shift}}\label{\detokenize{chapter2/2.2.6_Useful_Techniques::doc}}
\sphinxAtStartPar
Shift 은 이전 row 나 이후 row 에 있는 값을 가져올 수 있는 메소드입니다. 일단 삼성전자 일봉을 가져오겠습니다.

\begin{sphinxuseclass}{cell}\begin{sphinxVerbatimInput}

\begin{sphinxuseclass}{cell_input}
\begin{sphinxVerbatim}[commandchars=\\\{\}]
\PYG{k+kn}{import} \PYG{n+nn}{FinanceDataReader} \PYG{k}{as} \PYG{n+nn}{fdr} 

\PYG{n}{code} \PYG{o}{=} \PYG{l+s+s1}{\PYGZsq{}}\PYG{l+s+s1}{005930}\PYG{l+s+s1}{\PYGZsq{}} \PYG{c+c1}{\PYGZsh{} 삼성전자}
\PYG{n}{stock\PYGZus{}data} \PYG{o}{=} \PYG{n}{fdr}\PYG{o}{.}\PYG{n}{DataReader}\PYG{p}{(}\PYG{n}{code}\PYG{p}{,} \PYG{n}{start}\PYG{o}{=}\PYG{l+s+s1}{\PYGZsq{}}\PYG{l+s+s1}{2021\PYGZhy{}01\PYGZhy{}03}\PYG{l+s+s1}{\PYGZsq{}}\PYG{p}{,} \PYG{n}{end}\PYG{o}{=}\PYG{l+s+s1}{\PYGZsq{}}\PYG{l+s+s1}{2021\PYGZhy{}12\PYGZhy{}31}\PYG{l+s+s1}{\PYGZsq{}}\PYG{p}{)} 

\PYG{n}{stock\PYGZus{}data}\PYG{o}{.}\PYG{n}{head}\PYG{p}{(}\PYG{p}{)}\PYG{o}{.}\PYG{n}{style}\PYG{o}{.}\PYG{n}{set\PYGZus{}table\PYGZus{}attributes}\PYG{p}{(}\PYG{l+s+s1}{\PYGZsq{}}\PYG{l+s+s1}{style=}\PYG{l+s+s1}{\PYGZdq{}}\PYG{l+s+s1}{font\PYGZhy{}size: 12px}\PYG{l+s+s1}{\PYGZdq{}}\PYG{l+s+s1}{\PYGZsq{}}\PYG{p}{)}
\end{sphinxVerbatim}

\end{sphinxuseclass}\end{sphinxVerbatimInput}
\begin{sphinxVerbatimOutput}

\begin{sphinxuseclass}{cell_output}
\begin{sphinxVerbatim}[commandchars=\\\{\}]
\PYGZlt{}pandas.io.formats.style.Styler at 0x14e71565940\PYGZgt{}
\end{sphinxVerbatim}

\end{sphinxuseclass}\end{sphinxVerbatimOutput}

\end{sphinxuseclass}


\begin{sphinxuseclass}{cell}\begin{sphinxVerbatimInput}

\begin{sphinxuseclass}{cell_input}
\begin{sphinxVerbatim}[commandchars=\\\{\}]
\PYG{n}{stock\PYGZus{}data}\PYG{p}{[}\PYG{l+s+s1}{\PYGZsq{}}\PYG{l+s+s1}{Previous Close}\PYG{l+s+s1}{\PYGZsq{}}\PYG{p}{]} \PYG{o}{=} \PYG{n}{stock\PYGZus{}data}\PYG{p}{[}\PYG{l+s+s1}{\PYGZsq{}}\PYG{l+s+s1}{Close}\PYG{l+s+s1}{\PYGZsq{}}\PYG{p}{]}\PYG{o}{.}\PYG{n}{shift}\PYG{p}{(}\PYG{l+m+mi}{1}\PYG{p}{)}
\PYG{n}{stock\PYGZus{}data}\PYG{o}{.}\PYG{n}{head}\PYG{p}{(}\PYG{l+m+mi}{6}\PYG{p}{)}\PYG{o}{.}\PYG{n}{style}\PYG{o}{.}\PYG{n}{set\PYGZus{}table\PYGZus{}attributes}\PYG{p}{(}\PYG{l+s+s1}{\PYGZsq{}}\PYG{l+s+s1}{style=}\PYG{l+s+s1}{\PYGZdq{}}\PYG{l+s+s1}{font\PYGZhy{}size: 12px}\PYG{l+s+s1}{\PYGZdq{}}\PYG{l+s+s1}{\PYGZsq{}}\PYG{p}{)}
\end{sphinxVerbatim}

\end{sphinxuseclass}\end{sphinxVerbatimInput}
\begin{sphinxVerbatimOutput}

\begin{sphinxuseclass}{cell_output}
\begin{sphinxVerbatim}[commandchars=\\\{\}]
\PYGZlt{}pandas.io.formats.style.Styler at 0x14e6ec2af40\PYGZgt{}
\end{sphinxVerbatim}

\end{sphinxuseclass}\end{sphinxVerbatimOutput}

\end{sphinxuseclass}


\sphinxAtStartPar
그 다음 수익율 데이터를 만들어 보겠습니다. 내일의 시가는 stock\_data{[}‘Open’{]}.shift(\sphinxhyphen{}1), 내일의 종가는 stock\_data{[}‘Close’{]}.shift(\sphinxhyphen{}1) 로 가져오면 됩니다. 결과를 컬럼 ‘return’ 에 넣겠습니다. shift(1) 는 전날의 정보를 shift(\sphinxhyphen{}1) 은 다음날의 데이터를 가져옵니다.

\begin{sphinxuseclass}{cell}\begin{sphinxVerbatimInput}

\begin{sphinxuseclass}{cell_input}
\begin{sphinxVerbatim}[commandchars=\\\{\}]
\PYG{n}{stock\PYGZus{}data}\PYG{p}{[}\PYG{l+s+s1}{\PYGZsq{}}\PYG{l+s+s1}{buy}\PYG{l+s+s1}{\PYGZsq{}}\PYG{p}{]} \PYG{o}{=} \PYG{p}{(}\PYG{n}{stock\PYGZus{}data}\PYG{p}{[}\PYG{l+s+s1}{\PYGZsq{}}\PYG{l+s+s1}{Close}\PYG{l+s+s1}{\PYGZsq{}}\PYG{p}{]} \PYG{o}{\PYGZgt{}} \PYG{n}{stock\PYGZus{}data}\PYG{p}{[}\PYG{l+s+s1}{\PYGZsq{}}\PYG{l+s+s1}{Previous Close}\PYG{l+s+s1}{\PYGZsq{}}\PYG{p}{]}\PYG{p}{)}\PYG{o}{.}\PYG{n}{astype}\PYG{p}{(}\PYG{n+nb}{int}\PYG{p}{)} \PYG{c+c1}{\PYGZsh{} 매수 시그널 생성}
\PYG{n}{stock\PYGZus{}data}\PYG{p}{[}\PYG{l+s+s1}{\PYGZsq{}}\PYG{l+s+s1}{return}\PYG{l+s+s1}{\PYGZsq{}}\PYG{p}{]} \PYG{o}{=} \PYG{n}{stock\PYGZus{}data}\PYG{p}{[}\PYG{l+s+s1}{\PYGZsq{}}\PYG{l+s+s1}{Close}\PYG{l+s+s1}{\PYGZsq{}}\PYG{p}{]}\PYG{o}{.}\PYG{n}{shift}\PYG{p}{(}\PYG{o}{\PYGZhy{}}\PYG{l+m+mi}{1}\PYG{p}{)} \PYG{o}{/} \PYG{n}{stock\PYGZus{}data}\PYG{p}{[}\PYG{l+s+s1}{\PYGZsq{}}\PYG{l+s+s1}{Open}\PYG{l+s+s1}{\PYGZsq{}}\PYG{p}{]}\PYG{o}{.}\PYG{n}{shift}\PYG{p}{(}\PYG{o}{\PYGZhy{}}\PYG{l+m+mi}{1}\PYG{p}{)} \PYG{c+c1}{\PYGZsh{} 전략의 수익율}
\PYG{n}{stock\PYGZus{}data}\PYG{o}{.}\PYG{n}{head}\PYG{p}{(}\PYG{l+m+mi}{6}\PYG{p}{)}\PYG{o}{.}\PYG{n}{style}\PYG{o}{.}\PYG{n}{set\PYGZus{}table\PYGZus{}attributes}\PYG{p}{(}\PYG{l+s+s1}{\PYGZsq{}}\PYG{l+s+s1}{style=}\PYG{l+s+s1}{\PYGZdq{}}\PYG{l+s+s1}{font\PYGZhy{}size: 12px}\PYG{l+s+s1}{\PYGZdq{}}\PYG{l+s+s1}{\PYGZsq{}}\PYG{p}{)}
\end{sphinxVerbatim}

\end{sphinxuseclass}\end{sphinxVerbatimInput}
\begin{sphinxVerbatimOutput}

\begin{sphinxuseclass}{cell_output}
\begin{sphinxVerbatim}[commandchars=\\\{\}]
\PYGZlt{}pandas.io.formats.style.Styler at 0x14e6ece0f40\PYGZgt{}
\end{sphinxVerbatim}

\end{sphinxuseclass}\end{sphinxVerbatimOutput}

\end{sphinxuseclass}


\begin{sphinxuseclass}{cell}\begin{sphinxVerbatimInput}

\begin{sphinxuseclass}{cell_input}
\begin{sphinxVerbatim}[commandchars=\\\{\}]
\PYG{k+kn}{import} \PYG{n+nn}{pandas} \PYG{k}{as} \PYG{n+nn}{pd}
\PYG{n}{pd}\PYG{o}{.}\PYG{n}{options}\PYG{o}{.}\PYG{n}{display}\PYG{o}{.}\PYG{n}{float\PYGZus{}format} \PYG{o}{=} \PYG{l+s+s1}{\PYGZsq{}}\PYG{l+s+si}{\PYGZob{}:,.3f\PYGZcb{}}\PYG{l+s+s1}{\PYGZsq{}}\PYG{o}{.}\PYG{n}{format}

\PYG{n}{stock\PYGZus{}data}\PYG{o}{.}\PYG{n}{dropna}\PYG{p}{(}\PYG{n}{inplace}\PYG{o}{=}\PYG{k+kc}{True}\PYG{p}{)} \PYG{c+c1}{\PYGZsh{} NaN(값 없음) 열 전부 제거}
\PYG{n+nb}{print}\PYG{p}{(}\PYG{n}{stock\PYGZus{}data}\PYG{o}{.}\PYG{n}{groupby}\PYG{p}{(}\PYG{l+s+s1}{\PYGZsq{}}\PYG{l+s+s1}{buy}\PYG{l+s+s1}{\PYGZsq{}}\PYG{p}{)}\PYG{p}{[}\PYG{l+s+s1}{\PYGZsq{}}\PYG{l+s+s1}{return}\PYG{l+s+s1}{\PYGZsq{}}\PYG{p}{]}\PYG{o}{.}\PYG{n}{mean}\PYG{p}{(}\PYG{p}{)}\PYG{p}{)} \PYG{c+c1}{\PYGZsh{} 평균 비교}
\PYG{n+nb}{print}\PYG{p}{(}\PYG{l+s+s1}{\PYGZsq{}}\PYG{l+s+se}{\PYGZbs{}n}\PYG{l+s+s1}{\PYGZsq{}}\PYG{p}{)}
\PYG{n+nb}{print}\PYG{p}{(}\PYG{n}{stock\PYGZus{}data}\PYG{o}{.}\PYG{n}{groupby}\PYG{p}{(}\PYG{l+s+s1}{\PYGZsq{}}\PYG{l+s+s1}{buy}\PYG{l+s+s1}{\PYGZsq{}}\PYG{p}{)}\PYG{p}{[}\PYG{l+s+s1}{\PYGZsq{}}\PYG{l+s+s1}{return}\PYG{l+s+s1}{\PYGZsq{}}\PYG{p}{]}\PYG{o}{.}\PYG{n}{describe}\PYG{p}{(}\PYG{p}{)}\PYG{p}{)} \PYG{c+c1}{\PYGZsh{} 분포 비교}
\end{sphinxVerbatim}

\end{sphinxuseclass}\end{sphinxVerbatimInput}
\begin{sphinxVerbatimOutput}

\begin{sphinxuseclass}{cell_output}
\begin{sphinxVerbatim}[commandchars=\\\{\}]
buy
0   0.999
1   0.998
Name: return, dtype: float64


      count  mean   std   min   25\PYGZpc{}   50\PYGZpc{}   75\PYGZpc{}   max
buy                                                  
0   136.000 0.999 0.011 0.970 0.992 1.000 1.005 1.033
1   110.000 0.998 0.013 0.975 0.990 0.997 1.004 1.066
\end{sphinxVerbatim}

\end{sphinxuseclass}\end{sphinxVerbatimOutput}

\end{sphinxuseclass}


\begin{sphinxuseclass}{cell}\begin{sphinxVerbatimInput}

\begin{sphinxuseclass}{cell_input}
\begin{sphinxVerbatim}[commandchars=\\\{\}]
\PYG{n+nb}{print}\PYG{p}{(}\PYG{n}{stock\PYGZus{}data}\PYG{o}{.}\PYG{n}{groupby}\PYG{p}{(}\PYG{l+s+s1}{\PYGZsq{}}\PYG{l+s+s1}{buy}\PYG{l+s+s1}{\PYGZsq{}}\PYG{p}{)}\PYG{p}{[}\PYG{l+s+s1}{\PYGZsq{}}\PYG{l+s+s1}{return}\PYG{l+s+s1}{\PYGZsq{}}\PYG{p}{]}\PYG{o}{.}\PYG{n}{prod}\PYG{p}{(}\PYG{p}{)}\PYG{p}{)}
\end{sphinxVerbatim}

\end{sphinxuseclass}\end{sphinxVerbatimInput}
\begin{sphinxVerbatimOutput}

\begin{sphinxuseclass}{cell_output}
\begin{sphinxVerbatim}[commandchars=\\\{\}]
buy
0   0.843
1   0.811
Name: return, dtype: float64
\end{sphinxVerbatim}

\end{sphinxuseclass}\end{sphinxVerbatimOutput}

\end{sphinxuseclass}

\part{Rolling}
\label{\detokenize{chapter2/2.2.7_Useful_Techniques:rolling}}\label{\detokenize{chapter2/2.2.7_Useful_Techniques::doc}}
\sphinxAtStartPar
주식을 하신 분들은 이동평균선에 대하여 많이 들어보셨을 것이라고 생각합니다. rolling 은 이동평균선을 간단하게 만들어줄 수 있는 메소드입니다. 예제를 보시면 금방 이해가 되 실 것이라고 생각합니다. 일단 삼성전자 일봉을 가져오겠습니다.

\begin{sphinxuseclass}{cell}\begin{sphinxVerbatimInput}

\begin{sphinxuseclass}{cell_input}
\begin{sphinxVerbatim}[commandchars=\\\{\}]
\PYG{k+kn}{import} \PYG{n+nn}{FinanceDataReader} \PYG{k}{as} \PYG{n+nn}{fdr} 

\PYG{n}{code} \PYG{o}{=} \PYG{l+s+s1}{\PYGZsq{}}\PYG{l+s+s1}{005930}\PYG{l+s+s1}{\PYGZsq{}} \PYG{c+c1}{\PYGZsh{} 삼성전자}
\PYG{n}{stock\PYGZus{}data} \PYG{o}{=} \PYG{n}{fdr}\PYG{o}{.}\PYG{n}{DataReader}\PYG{p}{(}\PYG{n}{code}\PYG{p}{,} \PYG{n}{start}\PYG{o}{=}\PYG{l+s+s1}{\PYGZsq{}}\PYG{l+s+s1}{2021\PYGZhy{}01\PYGZhy{}03}\PYG{l+s+s1}{\PYGZsq{}}\PYG{p}{,} \PYG{n}{end}\PYG{o}{=}\PYG{l+s+s1}{\PYGZsq{}}\PYG{l+s+s1}{2021\PYGZhy{}12\PYGZhy{}31}\PYG{l+s+s1}{\PYGZsq{}}\PYG{p}{)} 

\PYG{n}{stock\PYGZus{}data}\PYG{o}{.}\PYG{n}{head}\PYG{p}{(}\PYG{p}{)}\PYG{o}{.}\PYG{n}{style}\PYG{o}{.}\PYG{n}{set\PYGZus{}table\PYGZus{}attributes}\PYG{p}{(}\PYG{l+s+s1}{\PYGZsq{}}\PYG{l+s+s1}{style=}\PYG{l+s+s1}{\PYGZdq{}}\PYG{l+s+s1}{font\PYGZhy{}size: 12px}\PYG{l+s+s1}{\PYGZdq{}}\PYG{l+s+s1}{\PYGZsq{}}\PYG{p}{)}
\end{sphinxVerbatim}

\end{sphinxuseclass}\end{sphinxVerbatimInput}
\begin{sphinxVerbatimOutput}

\begin{sphinxuseclass}{cell_output}
\begin{sphinxVerbatim}[commandchars=\\\{\}]
\PYGZlt{}pandas.io.formats.style.Styler at 0x164cee62640\PYGZgt{}
\end{sphinxVerbatim}

\end{sphinxuseclass}\end{sphinxVerbatimOutput}

\end{sphinxuseclass}


\begin{sphinxuseclass}{cell}\begin{sphinxVerbatimInput}

\begin{sphinxuseclass}{cell_input}
\begin{sphinxVerbatim}[commandchars=\\\{\}]
\PYG{n}{stock\PYGZus{}data}\PYG{p}{[}\PYG{l+s+s1}{\PYGZsq{}}\PYG{l+s+s1}{5 day moving average}\PYG{l+s+s1}{\PYGZsq{}}\PYG{p}{]} \PYG{o}{=} \PYG{n}{stock\PYGZus{}data}\PYG{p}{[}\PYG{l+s+s1}{\PYGZsq{}}\PYG{l+s+s1}{Close}\PYG{l+s+s1}{\PYGZsq{}}\PYG{p}{]}\PYG{o}{.}\PYG{n}{rolling}\PYG{p}{(}\PYG{l+m+mi}{5}\PYG{p}{)}\PYG{o}{.}\PYG{n}{mean}\PYG{p}{(}\PYG{p}{)}
\PYG{n}{stock\PYGZus{}data}\PYG{o}{.}\PYG{n}{head}\PYG{p}{(}\PYG{l+m+mi}{6}\PYG{p}{)}\PYG{o}{.}\PYG{n}{style}\PYG{o}{.}\PYG{n}{set\PYGZus{}table\PYGZus{}attributes}\PYG{p}{(}\PYG{l+s+s1}{\PYGZsq{}}\PYG{l+s+s1}{style=}\PYG{l+s+s1}{\PYGZdq{}}\PYG{l+s+s1}{font\PYGZhy{}size: 12px}\PYG{l+s+s1}{\PYGZdq{}}\PYG{l+s+s1}{\PYGZsq{}}\PYG{p}{)}
\end{sphinxVerbatim}

\end{sphinxuseclass}\end{sphinxVerbatimInput}
\begin{sphinxVerbatimOutput}

\begin{sphinxuseclass}{cell_output}
\begin{sphinxVerbatim}[commandchars=\\\{\}]
\PYGZlt{}pandas.io.formats.style.Styler at 0x164cc568880\PYGZgt{}
\end{sphinxVerbatim}

\end{sphinxuseclass}\end{sphinxVerbatimOutput}

\end{sphinxuseclass}


\begin{sphinxuseclass}{cell}\begin{sphinxVerbatimInput}

\begin{sphinxuseclass}{cell_input}
\begin{sphinxVerbatim}[commandchars=\\\{\}]
\PYG{n}{stock\PYGZus{}data}\PYG{p}{[}\PYG{l+s+s1}{\PYGZsq{}}\PYG{l+s+s1}{20 day moving average}\PYG{l+s+s1}{\PYGZsq{}}\PYG{p}{]} \PYG{o}{=} \PYG{n}{stock\PYGZus{}data}\PYG{p}{[}\PYG{l+s+s1}{\PYGZsq{}}\PYG{l+s+s1}{Close}\PYG{l+s+s1}{\PYGZsq{}}\PYG{p}{]}\PYG{o}{.}\PYG{n}{rolling}\PYG{p}{(}\PYG{l+m+mi}{20}\PYG{p}{)}\PYG{o}{.}\PYG{n}{mean}\PYG{p}{(}\PYG{p}{)}
\PYG{n}{stock\PYGZus{}data}\PYG{o}{.}\PYG{n}{head}\PYG{p}{(}\PYG{l+m+mi}{21}\PYG{p}{)}\PYG{o}{.}\PYG{n}{style}\PYG{o}{.}\PYG{n}{set\PYGZus{}table\PYGZus{}attributes}\PYG{p}{(}\PYG{l+s+s1}{\PYGZsq{}}\PYG{l+s+s1}{style=}\PYG{l+s+s1}{\PYGZdq{}}\PYG{l+s+s1}{font\PYGZhy{}size: 12px}\PYG{l+s+s1}{\PYGZdq{}}\PYG{l+s+s1}{\PYGZsq{}}\PYG{p}{)}
\end{sphinxVerbatim}

\end{sphinxuseclass}\end{sphinxVerbatimInput}
\begin{sphinxVerbatimOutput}

\begin{sphinxuseclass}{cell_output}
\begin{sphinxVerbatim}[commandchars=\\\{\}]
\PYGZlt{}pandas.io.formats.style.Styler at 0x164cfdd35e0\PYGZgt{}
\end{sphinxVerbatim}

\end{sphinxuseclass}\end{sphinxVerbatimOutput}

\end{sphinxuseclass}


\begin{sphinxuseclass}{cell}\begin{sphinxVerbatimInput}

\begin{sphinxuseclass}{cell_input}
\begin{sphinxVerbatim}[commandchars=\\\{\}]
\PYG{n}{stock\PYGZus{}data}\PYG{o}{.}\PYG{n}{dropna}\PYG{p}{(}\PYG{n}{inplace}\PYG{o}{=}\PYG{k+kc}{True}\PYG{p}{)} \PYG{c+c1}{\PYGZsh{} NaN 이 있는 모든 row 제거}
\PYG{n}{stock\PYGZus{}data}\PYG{o}{.}\PYG{n}{head}\PYG{p}{(}\PYG{p}{)}\PYG{o}{.}\PYG{n}{style}\PYG{o}{.}\PYG{n}{set\PYGZus{}table\PYGZus{}attributes}\PYG{p}{(}\PYG{l+s+s1}{\PYGZsq{}}\PYG{l+s+s1}{style=}\PYG{l+s+s1}{\PYGZdq{}}\PYG{l+s+s1}{font\PYGZhy{}size: 12px}\PYG{l+s+s1}{\PYGZdq{}}\PYG{l+s+s1}{\PYGZsq{}}\PYG{p}{)}
\end{sphinxVerbatim}

\end{sphinxuseclass}\end{sphinxVerbatimInput}
\begin{sphinxVerbatimOutput}

\begin{sphinxuseclass}{cell_output}
\begin{sphinxVerbatim}[commandchars=\\\{\}]
\PYGZlt{}pandas.io.formats.style.Styler at 0x164cf076310\PYGZgt{}
\end{sphinxVerbatim}

\end{sphinxuseclass}\end{sphinxVerbatimOutput}

\end{sphinxuseclass}


\begin{sphinxuseclass}{cell}\begin{sphinxVerbatimInput}

\begin{sphinxuseclass}{cell_input}
\begin{sphinxVerbatim}[commandchars=\\\{\}]
\PYG{n}{stock\PYGZus{}data}\PYG{p}{[}\PYG{l+s+s1}{\PYGZsq{}}\PYG{l+s+s1}{cross\PYGZus{}flag}\PYG{l+s+s1}{\PYGZsq{}}\PYG{p}{]} \PYG{o}{=} \PYG{p}{(}\PYG{n}{stock\PYGZus{}data}\PYG{p}{[}\PYG{l+s+s1}{\PYGZsq{}}\PYG{l+s+s1}{5 day moving average}\PYG{l+s+s1}{\PYGZsq{}}\PYG{p}{]} \PYG{o}{\PYGZgt{}} \PYG{n}{stock\PYGZus{}data}\PYG{p}{[}\PYG{l+s+s1}{\PYGZsq{}}\PYG{l+s+s1}{20 day moving average}\PYG{l+s+s1}{\PYGZsq{}}\PYG{p}{]}\PYG{p}{)}\PYG{o}{.}\PYG{n}{astype}\PYG{p}{(}\PYG{n+nb}{int}\PYG{p}{)} \PYG{c+c1}{\PYGZsh{} True/False 결과 값을 1/0 으로 바꿔줌}
\PYG{n}{s} \PYG{o}{=} \PYG{n}{stock\PYGZus{}data}\PYG{p}{[}\PYG{p}{(}\PYG{n}{stock\PYGZus{}data}\PYG{p}{[}\PYG{l+s+s1}{\PYGZsq{}}\PYG{l+s+s1}{cross\PYGZus{}flag}\PYG{l+s+s1}{\PYGZsq{}}\PYG{p}{]}\PYG{o}{.}\PYG{n}{shift}\PYG{p}{(}\PYG{l+m+mi}{1}\PYG{p}{)}\PYG{o}{==}\PYG{l+m+mi}{0}\PYG{p}{)} \PYG{o}{\PYGZam{}} \PYG{p}{(}\PYG{n}{stock\PYGZus{}data}\PYG{p}{[}\PYG{l+s+s1}{\PYGZsq{}}\PYG{l+s+s1}{cross\PYGZus{}flag}\PYG{l+s+s1}{\PYGZsq{}}\PYG{p}{]}\PYG{o}{==}\PYG{l+m+mi}{1}\PYG{p}{)}\PYG{p}{]} \PYG{c+c1}{\PYGZsh{} 조건 \PYGZhy{} 전날에는 5일 이평선이 20일 이평선보다 작거나 같아는데, 당일은 5일 이평선이 20일 이평선 보다 커짐}
\PYG{n}{s}\PYG{o}{.}\PYG{n}{style}\PYG{o}{.}\PYG{n}{set\PYGZus{}table\PYGZus{}attributes}\PYG{p}{(}\PYG{l+s+s1}{\PYGZsq{}}\PYG{l+s+s1}{style=}\PYG{l+s+s1}{\PYGZdq{}}\PYG{l+s+s1}{font\PYGZhy{}size: 12px}\PYG{l+s+s1}{\PYGZdq{}}\PYG{l+s+s1}{\PYGZsq{}}\PYG{p}{)}
\end{sphinxVerbatim}

\end{sphinxuseclass}\end{sphinxVerbatimInput}
\begin{sphinxVerbatimOutput}

\begin{sphinxuseclass}{cell_output}
\begin{sphinxVerbatim}[commandchars=\\\{\}]
\PYGZlt{}pandas.io.formats.style.Styler at 0x164cee62400\PYGZgt{}
\end{sphinxVerbatim}

\end{sphinxuseclass}\end{sphinxVerbatimOutput}

\end{sphinxuseclass}






\renewcommand{\indexname}{Index}
\printindex
\end{document}